\section{Язык и речь.}

Слово «речь» обозначает специфическую человеческую деятельность, поэтому для характеристики обеих ее сторон это слово в лингвистике употребляется в двух основных значениях: \textbf{речью} называют и сам \textbf{процесс говорения} (в устной форме) или \textbf{письма} (в письменной), и те речевые произведения (высказывания, устные и письменные тексты), которые представляют собой звуковой или графический \textbf{продукт (результат)} этой деятельности.

Язык и речь теснейшим образом взаимосвязаны, поскольку речь --- это язык в действии, и для достижения высокой культуры речи язык и речь необходимо различать.

Прежде всего, \textbf{язык --- это система знаков,} а \textbf{речь --- это деятельность,} протекающая как процесс и представленная как продукт этой деятельности.
И хотя речь строится на том или ином языке, это самое главное отличие, которое по различным основаниям определяет и другие.

\textbf{Речь} представляет собой способ реализации всех функций языка, прежде всего — коммуникативной.
Речь возникает как необходимый ответ на те или иные события действительности (в том числе и речевые), поэтому она, в отличие от языка, \textbf{преднамеренна} и \textbf{ориентирована на определенную цель.}

\textbf{Речь,} прежде всего, \textbf{материальна} — в устной форме она звучит, а в письменной она фиксируется с помощью соответствующих графических средств (иногда отличных отданного языка, например, в другой графической системе (латинице, кириллице, иероглифическом письме) или с помощью значков, формул, рисунков и пр.).
\textbf{Речь зависит от конкретных ситуаций, развертывается во времени и реализуется в пространстве.}
Например, вaш ответ по одному из предметов, которые вы изучаете, будет по-разному строиться в зависимости от того, насколько вам знаком этот материал, насколько он сложен, как долго вы можете говорить или сколько времени у вас есть на подготовку, в каком помещении и на каком расстоянии от адресата речи вы будете ее произносить и~т.~д.
Речь создается конкретным человеком в конкретных условиях, для конкретного человека (аудитории), следовательно, она всегда \textbf{конкретна} и \textbf{неповторима,} потому что, даже если она воспроизводится с помощью тех или иных записей, меняются обстоятельства и получается то же самое, про что обычно говорят: «Нельзя дважды войти в одну и ту же реку».
При этом теоретически \textbf{речь может длиться бесконечно долго} (с перерывами и без них).
По сути, вся наша жизнь с тех пор, как мы начинаем говорить и до тех пор, когда скажем последнее слово, --- это одна большая речь, в которой меняются обстоятельства, адресат, предмет речи, форма (устная или письменная) и~пр., но мы продолжаем говорить (или писать).
И с нашим последним словом речь (только уже письменная или не наша устная) будет продолжаться.
В этом плане речь \textbf{развертывается линейно,} то есть мы произносим одно предложение за другим в определенной последовательности.
Процесс устной речи характеризуется тем, что \textbf{речь протекает в определенном (иногда меняющемся) темпе, с большей или меньшей продолжительностью, степенью громкости, артикуляционной четкости} и~т.~п.
Письменная речь также может быть быстрой или медленной, четкой (разборчивой) или нечеткой (неразборчивой), более или менее объемной и~т.~д.
То есть материальность речи можно иллюстрировать разными примерами.
Язык же, в отличие от речи, как считается, идеален, то есть он существует вне речи как целое только в сознании говорящих на этом языке или изучающих этот язык, а также как части этого целого — в различных словарях и справочниках.

\textbf{Речь} представляет собой, как правило, деятельность одного человека — говорящего или пишущего, поэтому она является отражением разнообразных особенностей этого человека.
Следовательно, речь изначально \textbf{субъективна,} потому что говорящий или пишущий сам отбирает содержание своей речи, отражает в ней свое индивидуальное сознание и индивидуальный опыт, язык же в системе выражаемых им значений фиксирует опыт коллектива, «картину мира» говорящего на нем народа.
Кроме того, речь всегда индивидуальна, поскольку люди никогда не используют все средства языка и довольствуются лишь частью языковых средств, выбирая наиболее подходящие сообразно своему уровню знаний языка и условиям конкретной ситуации.
Вследствие этого значения слов в речи могут расходиться с теми, что строго определены и зафиксированы словарями.
В речи возможны ситуации, в которых слова и даже отдельные предложения получают совсем другой смысл, чем в языке, например, с помощью интонации.
Речь может быть охарактеризована и через указание на психологическое состояние говорящего, его коммуникативную задачу, отношение к собеседнику, искренность.

Различие языка и речи можно увидеть, как на своеобразном срезе, в сопоставлении \textbf{предложения как единицы языка и высказывания как единицы речи.}
М. М. Бахтин так разграничивает эти понятия: «Предложение как единица языка имеет грамматическую природу, грамматические границы, грамматическую законченность и единство.
(Рассматриваемое в целом высказывания и с точки зрения этого целого, оно приобретает стилистические свойства)».
«Предложение как единица языка <...> не отграничивается с обеих сторон сменой речевых субъектов, оно не имеет непосредственного контакта с действительностью (с внесловесной ситуацией) и непосредственного же отношения к чужим высказываниям, оно не обладает смысловой полноценностью и способностью непосредственно определять ответную позицию \textit{другого} говорящего, то есть вызывать ответ».
В свою очередь, высказывание отличается от предложения тем, что оно всегда связано с речевой ситуацией, ориентировано не только на чье-то чужое высказывание, но и на наличие адресата и на его активную ответную позицию, высказывание также обладает смысловой завершенностью и четко выраженными границами между высказываниями других речевых субъектов.

Кроме того, \textbf{речь не ограничивается только языковыми средствами.}
\textbf{В состав речевых средств входят также} те, что относятся
к \textbf{неязыковым} (несловесным, или невербальным): голос, интонация, жесты, мимика, поза, положение в пространстве и т. д.

Все эти отличия речи от языка относятся прежде всего к речи как процессу использования языка, поэтому, хотя и с натяжкой, являются основаниями для их противопоставления, поскольку в этом плане создание речи как процесс протекает во многом поэтапно и частично совпадает с границами самой большой единицы языка: с границами предложения.
Если же говорить о речи как результате этого процесса, то есть как о тексте, то описание речи на этом уровне в принципе не может иметь общих критериев с языком, поскольку они к языку совершенно неприменимы.
А именно:

\textbf{Речь} бывает \textbf{внешней} (произнесенной или написанной) и \textbf{внутренней} (не озвученной и не зафиксированной для других).
Внутренняя речь используется нами как средство мышления или внутреннего проговаривания (речь минус звук), а также как способ запоминания.

\textbf{Речь-высказывание} протекает в определенных \textbf{речевых жанрах,} например, письмо, выступление, прощание и т. д.

\textbf{Речь-текст} должна строиться в соответствии с тем или иным \textbf{функциональным стилем:} научным, официально-деловым, публицистическим, разговорным или художественным.

\textbf{Речь как текст} отражает действительность и может рассматриваться с точки зрения своей истинности и ложности (истинно / частично истинно / ложно).

\textbf{К речи-тексту} применимы \textbf{эстетические} (красиво / некрасиво / безобразно) \textbf{и этические оценки} (хорошо / плохо) и т. д.

Таким образом, мы видим, что все функции языка реализуются в речи.
И язык оказывается главным, но не единственным средством ее создания.
Речь всегда представляет собой результат творческой деятельности индивида, поэтому и подходить к анализу, оценке и способам создания речи нужно совсем иначе, чем к языку.
Особенно это важно при рассмотрении речи с точки зрения ее культуры.
