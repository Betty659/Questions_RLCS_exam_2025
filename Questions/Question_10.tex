\section{Литературный язык. Этапы становления русского литературного языка.}
 

\textbf{Русский национальный язык} --- это общенародный русский язык.
Он охватывает все сферы речевой деятельности, независимо от образования, воспитания, территории проживания индивида, и включает в себя литературный язык с его функциональными стилями, диалекты, профессиональные жаргоны и просторечие.
\textbf{\\ Литературный язык } — это основная форма языка, которая характеризуется обработанностью, многофункциональность, отражает стилистические особенности той или иной сферы общения, и что нужно отметить особо,
обладает нормативностью. Этим литературный язык отличается от всех остальных разновидностей языка. При этом литературный язык охватывает все основные сферы общения: повседневную(бытовую), научную, официально-деловую, публичную
и сферу искусства слова. И во всех этих сферах литературный язык не только обеспечивает взаимопонимание, но и повышает
общий уровень культуры, помогает достичь большей эффективности речи путем использования как общелитературных, так и
специфических для данной сферы языковых средств. Это находит отражение в разветвленной системе функциональных стилей
русского языка, соответствующих основным сферам общения. Литературный язык называется так потому, что в основе его
создания лежит отбор всего лучшего, что есть в языке и что нуждается в сохранении и развитии, то есть культура языка.
\textbf{Русский литературный язык} --- понятие более узкое. Это язык, обработанный мастерами слова: писателями, общественными
деятелями, учеными.
Под современным русским литературным языком понимается состояние языка в период с первой трети XIX в. до наших дней.
Выдающаяся роль в создании современного русского литературного языка принадлежит А.С. Пушкину, который считал язык «первым училищем для юной души».
Его произведения открыли целую эпоху в развитии нового русского литературного языка.

\subsection*{Формы русского литературного языка.}

Современный русский литературный язык существует в двух формах: книжно-письменной и устно-разговорной, которые обладают определенным набором признаков.

\textbf{Книжно-письменную форму} отличают общепризнанность, определенность, продуманность, правильность и редакторская обработка.
Для письменной речи характерно следующее:

\begin{itemize}
    \item сложная система графики, орфографии и пунктуации;
    \item строгое соблюдение литературных норм;
    \item тщательный отбор лексики и фразеологии;
    \item употребление сложных и осложненных предложений;
    \item особая роль порядка слов;
    \item монологическая форма.
\end{itemize}

Главная характеристика \textbf{устно-разговорной формы} - спонтанность. Кроме этого, отмечаются следующие признаки:

\begin{itemize}
    \item смыслообразующая роль интонации;
    \item наличие просторечной лексики и фразеологии;
    \item редкое использование причастных и деепричастных оборотов, сложноподчиненных предложений с разнообразными типами связи;
    \item диалогическая форма;
    \item широкое применение паралингвистических средств: мимики, жестов.
\end{itemize}

\subsection*{Этапы становления русского литературного языка.}

Алфавит, которым мы сейчас пользуемся, называется \textbf{кириллицей.}
В IX в. (863 г.) монахи Кирилл и Мефодий по указанию византийского императора создали славянскую азбуку и с целью распространения христианства перевели первые греческие богослужебные тексты на славянский язык.
В основу старославянского языка (так именуется язык первых переводов) был положен один из диалектов македонского языка.
Старославянский называют \textit{мертвым} языком: на нем никто никогда не говорил, потому что славянские племена, получившие от миссионеров тексты на старославянском языке, были носителями различных славянских диалектов.

Кириллическую азбуку начали использовать уже в Древнем Новгороде.
В ходе археологических раскопок были обнаружены берестяные грамоты --- записки бытового содержания жителей Новгорода.
Эти находки, относящиеся к IX в., свидетельствуют о начале распространения грамотности на Руси.

Все буквы старославянской азбуки имели особое название: \textit{А --- аз, Б --- буки, В --- веди, Г --- глаголь, Д --- добро, Е --- есть, Ж --- живете, 3 --- зело, I --- иже, К --- како, Л --- людие, М --- мыслете, Н --- наш, О --- он, П --- покои, Р --- рцы, С --- слово, Т --- твердо} и т.д.

Для передачи греческих заимствований использовались греческие буквы: \textit{$\mit\Psi$ --- пси, $\mit\Xi$ --- кси, $\mit \Theta$ --- фита.}

В азбуке были и забавные названия букв: \textit{X --- хер, Ф --- ферт.}

Отсюда происходят глагол \textit{похерить} --- ‘перечеркнуть, запретить’ и выражение \textit{ходить, стоять фертом,} то есть подбоченясь.
Со словесным обозначением букв \textit{X} и \textit{О} связано наименование дореволюционной гимназической игры --- \textit{херики и оники} (современное название --- \textit{крестики и нолики}).
Почти все буквы кириллицы, за исключением \textit{Б, Ж} и некоторых других, имели числовое значение (в этом случае над буквой ставился знак титло ~). Например, \~{А} --- 1, \~{I} --- 10, \~{К} --- 20, \~{Л} --- 5O, \~{П} --- 80, \~{Т} --- 300; \~{ГI} --- 13 («три на десяте») и т.д.

Буквы под титлом использовались для числовых обозначений и расчетов вплоть до начала XVIII в., что создавало значительные неудобства для развития науки и международных торговых связей.
В 1702 г. Петр I, после взятия крепости Орешек (Нотебург), повелел издать журнал, в котором были представлены арабские цифры с числовыми значениями.
А в 1703 г. была напечатана первая учебная книга с арабскими цифрами --- «Арифметика» Магницкого, учителя Петра I Леонтия Теляшина, которому остроумный ученик дал прозвище «магнит».

Была и другая славянская азбука, именуемая \textbf{глаголицей.} Начертания букв в глаголице совершенно не похожи на привычные для нас формы.
Вопрос о первичности происхождения кириллицы и глаголицы является дискуссионным.Некоторые ученые считают глаголицу своеобразной тайнописью, которая появилась в период нашествия врагов православия и угрозы уничтожения письменных святынь.

В образовании литературного языка большую роль сыграл московский приказный язык XVI-XVII в. --- административно-деловой по происхождению язык московских канцелярий (приказов), в которых профессиональные писцы составляли тексты по установленным образцам-формулярам.
Приказный язык постепенно расширял свои функции (в XVII в. появились переводы на приказном языке, сложилась «приказная школа» стихотворства) и в результате стал ядром нового русского литературного языка.

Принято считать, что письменность является ключевым условием прогресса.
Однако эта точка зрения разделяется не всеми.
Последние наблюдения физиологов из UCL (University College of London) свидетельствуют о том, что способность говорить генетически обусловлена, в отличие от способности читать, которая не заложена в наших генах.
Человеку приходится обучать свой мозг раскодированию письменных знаков.
Ученые установили, что отделы мозга, которые управляют памятью и интерпретацией зрительной и звуковой информации, работают по-разному.
И записанный текст не всегда становится достоянием памяти --- хранителя знаний.
\begin{center}

\end{center}
Современный литературный язык - это строго нормированная и кодифицированная форма общенародного национального языка.
