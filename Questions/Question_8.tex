\section{Фамильярно-разговорная речевая культура.}

\begin{itemize}
    \item среднее, профессиональное среднее, средне-техническое, иногда высшее техническое образование;
    \item доминирует сенсомоторный интеллект, работа не требует систематических интеллектуальных усилий;
    \item низкое стремление к расширению общих знаний;
    \item доминирует сенсомоторный тип интеллекта;
    \item владение только разговорной системой общения, которая используется в любой обстановке, в том числе и официальной;
    \item неразличение норм устной и письменной форм речи;
    \item несоблюдение этических и коммуникативных норм в профессиональных ситуациях и межличностном общении;
    \item отсутствие стремления расширять языковые знания, удовлетворенность своими языковыми знаниями;
    \item отсутствие привычки узнавать значения слов или правила их употребления, языковая самоуверенность;
    \item доминирование точки зрения «главное, ЧТО сказать, а не КАК сказать»;
    \item ориентация на языковую среду как единственный критерий языковой нормы;
    \item небольшой словарный запас;
    \item сбивчивость, нелогичность речи, ориентированной только на диалогическую форму;
    \item неумение строить сколько-нибудь развернутый связный монологический текст, даже подготовленный;
    \item распространение законов непринужденного персонально адресованного неофициального общения на любые коммуникативные ситуации;
    \item коммуникативная беспомощность в официальных ситуациях;
    \item трудности в ситуации письменных форм коммуникации, потребность в образце для написания текста по аналогии;
    \item прецедентными текстами являются преимущественно рекламные тексты, в основном --- теле- и стендовая реклама;
    \item неспособность к чтению более или менее длинных текстов любого жанра, плохо синтезируется смысл текста;
    \item любовь к кроссвордам и комиксам, газетам с анекдотами и короткими иллюстрированными заметками;
    \item неумение и нежелание пользоваться словарями, уклонение от их использования;
    \item преобладание \textit{ты}-общения;
    \item преимущественное использование в обращении единиц типа \textit{Сережка}, \textit{Серега}, \textit{Михалыч}, \textit{Петрович} и под.;
    \item повышенная громкость речи;
    \item преимущественное использование разговорного (неполного) стиля произношения --- скороговорка с предельной редукцией;
    \item фамильярная фразовая интонация;
    \item отсутствие контроля за громкостью, интонацией речи;
    \item большая доля грубых слов и просторечных элементов в речи;
    \item большое количество используемых в речи жаргонизмов, иноязычной лексики, книжных, бюрократических слов, которые такие люди слышат вокруг и употребляют их, не понимая их стилистической окраски, уместности и под., использование таких слов просто потому что они модные или часто употребляемые другими; такие слова в их речи нередко становятся простыми заполнителями пауз;
    \item частотность заполнения пауз словами \textit{конкретно}, \textit{короче}, \textit{типа}, \textit{в натуре}, \textit{блин}, \textit{бля} и т. п.;
    \item преобладание «речевого» юмора, построенного на употреблении сниженной лексики, непонимание юмора, основанного на смысловом подтексте;
    \item погоня за языковой модой, тяга к модным экспрессивным словоупотреблениям;
    \item неспособность к синонимическому варьированию речи, что приводит к штампованности и отсутствию индивидуальности в речи;
    \item экспрессия речи достигается в основном использованием категоричных оценок, сниженной лексики, повышением громкости или интонационной напряженностью артикуляции;
    \item отсутствие привычки обсуждать проблемы языка в какой-либо форме.
\end{itemize}
