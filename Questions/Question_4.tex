\section{Типы речевой культуры: общая характеристика.}

В современной лингвистике утвердилась типология внутринациональных речевых культур, которые сосуществуют в деловом общении и непосредственно связаны с образовательным и культурным уровнем говорящих и пишущих.

Самым высоким типом речевой культуры является \textbf{элитарный тип.}
Речь представителя элитарной речевой культуры не только безукоризненна с точки зрения соблюдения языковых норм, но и отличается богатством словарного запаса, выразительностью, аргументированностью, логичностью, доступностью и ясностью изложения.

\textbf{Среднелитературному типу} речевой культуры свойственна меньшая строгость соблюдения всех норм, при этом ошибки в устной и письменной речи представителей этого типа речевой культуры не носят систематического характера.
Данный тип речевой культуры характеризуется некоторым смешением норм устной и письменной речи: иногда в устной речи используются книжные штампы, причастные или деепричастные обороты, а в письменную речь (в частности, в язык документов) проникают разговорные конструкции и жаргонизмы.

Среднелитературный тип речевой культуры отличается нестрогим выполнением этикетных требований: обращением на \textit{ты} при каждом удобном случае, низкой частотностью использования этикетных форм, которые обычно представлены очень ограниченным набором: \textit{спасибо, здравствуйте, до свидания, извините.}

Если носители элитарной речевой культуры оперируют всеми стилями, то представители среднелитературной речевой культуры обычно используют лишь один-два стиля (например, официально-деловой и разговорный), остальными же владеют только пассивно.

Хотя, в отличие от элитарной, среднелитературная речевая культура не является эталонной, однако этот тип речевой культуры самый массовый во всех сферах нашей общественной жизни и представляет речь большинства теле- и радиожурналистов, поэтому речевые ошибки, к сожалению, тиражируются: \textit{кв\underline{а}ртал, в\underline{а}ловый, \underline{э}ксперт, обеспеч\underline{е}ние, отзвонить, отследить, разговор по экономике, расчет по плитам, отмечая о том, что...}~и~т.п.

С точки зрения принадлежности к речевым культурам, деловые люди чаще всего являются представителями среднелитературного типа речевой культуры.

\textbf{Разговорный,} или \textbf{фамильярно-разговорный, тип} речевой культуры может быть разновидностью элитарного и среднелитературного типа речевой культуры, если общение протекает в неофициальной обстановке, в сфере близкородственного, дружеского общения.
Этот тип речевой культуры допускает использование в узкой корпоративной среде сниженной лексики (жаргонизмов, просторечных выражений, обсценной лексики) при общем соблюдении языковых норм.

К еще более низким типам культур относится \textbf{просторечный тип.}
Просторечие является показателем низкого образовательного и культурного уровня.
Носитель просторечия отличается ограниченным запасом слов, неумением строить сложные предложения; его речь характеризуется высокой частотностью экспрессивных слов, ругательств, слов-паразитов, междометий.

Ниже приводятся развёрнутые характеристики представителей каждого типа речевой культуры.
