
\section{Правила оформления презентации для публичного выступления}

Презентация — важный элемент публичного выступления. Она должна помогать, а не мешать восприятию основного материала. В этом документе изложены ключевые правила, которых следует придерживаться при её оформлении.

\section*{Общие рекомендации}

\begin{itemize}
	\item Презентация должна быть лаконичной и визуально опрятной.
	\item Избегайте перегруженности текста и чрезмерного количества слайдов.
	\item Используйте крупный и читаемый шрифт, предпочтительно без засечек.
	\item Поддерживайте единый стиль: цвета, шрифты и оформление заголовков.
\end{itemize}

\section*{Цветовая гамма и фон}

\begin{itemize}
	\item Используйте контрастные цвета для текста и фона.
	\item Избегайте ярких или кислотных оттенков.
	\item Белый или тёмно-серый фон с чёрным или тёмным текстом — классическое решение.
\end{itemize}

\section*{Графика и визуализация}

\begin{itemize}
	\item Иллюстрации должны быть качественными и уместными.
	\item Графики и диаграммы следует сопровождать пояснительным текстом.
	\item Избегайте излишнего анимационного оформления.
\end{itemize}

\section*{Текст на слайдах}

\begin{itemize}
	\item Используйте короткие тезисы, а не полные абзацы.
	\item Один слайд — одна мысль.
	\item Заголовки должны чётко отражать содержание слайда.
\end{itemize}

\section*{Структура презентации}

\begin{enumerate}
	\item Титульный слайд: название, имя, дата, место.
	\item Содержание: поэтапное раскрытие темы.
	\item Заключение: подведение итогов, основные выводы.
	\item Контактная информация.
\end{enumerate}

\section*{Заключение}

Хорошо оформленная презентация усиливает эффект от публичного выступления. Следуя простым и сдержанным правилам, можно добиться профессионального визуального сопровождения речи, не отвлекая аудиторию от содержания.
