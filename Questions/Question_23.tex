\section{Чтение как вид речевой деятельности.}


\textbf{Чтение -- вид речевой деятельности, в основе которого лежит восприятие и осмысление письменного текста. }

Чтение связано с переводом буквенного кода в звуковой либо во внешней (громкое чтение, чтение вслух), либо во внутренней (чтение про себя) речи.
Характерной особенностью чтения является \textit{осмысление зрительно воспринимаемого текста} с целью решения определенной коммуникативной задачи: распознавание и воспроизведение чужой мысли. 
Следовательно, с помощью чтения человек реализует возможности так называемого \textbf{опосредованного общения}: восприятие и понимание текста свидетельствует о взаимодействии читателя с автором текста.

Умение читать предполагает овладение техникой чтения, то есть правильным озвучиванием текста, записанного в определенной графической системе, и умением осмыслить, интерпретировать, понять прочитанное.

Как и любой другой вид речевой деятельности, чтение связано с решением определенной коммуникативной задачи.
Под коммуникативной задачей в данном случае следует понимать установку на то, с какой целью осуществляется чтение: \textbf{где, когда, для чего будет использована извлеченная из текста информация}.
Еще одна функция чтения условно может быть названа \textbf{функцией реагирования}, которая реализуется в процессе \textbf{критического осмысления информации}

\textbf{
\begin{center}
		Какие виды чтения существуют?
\end{center}}

Так как в процессе чтения решаются различные коммуникативные задачи, то реализуются разные виды чтения. 
В литературе, посвященной проблемам чтения, выделяется разное количество видов чтения: \textbf{изучающее, ознакомительное, просмотровое, аналитическое, выборочное, быстрое, медленное}.

\textbf{Ознакомительное чтение} предполагает беглое просматривание, фрагментарное, избирательное прочитывание текста, чтобы выявить в самом общем виде его характер: о чем в нем говорится, кому он адресован, насколько полно в нем, судя по оглавлению, аннотаций и другим признакам, освещена та или иная проблема.

\textbf{Изучающее чтение}, как правило, предполагает воспроизведение прочитанного в определенном объеме, что обусловлено необходимостью на основе прочитанного подготовиться к занятиям, зачету, экзамену, принять участие в научной дискуссии и т.~п.