
\section{Виды речевой деятельности. Общая характеристика.}

Речевая деятельность представляет собой процесс и является самым распространенным видом человеческой
деятельности.
Существуют 4 вида речевой деятельности:
\begin{itemize}
	\item аудирование (слушание);
	\item говорение;
	\item чтение;
	\item письмо.
\end{itemize}
На две трети человеческая деятельность состоит из речевой.
Особенность речевой деятельности заключается в том, что она всегда включается в более широкую систему деятельности как необходимый компонент.
Речевая деятельность имеет сознательный характер.

\subsection{Слушание}
Слушание представляет собой вид речевой деятельности, при котором происходит одновременное восприятие и понимание звучащей речи.

Виды слушания:
\begin{itemize}
	\item \textbf{Нерефлексивным} называется такой вид слушания, который
	не предполагает выраженной реакции на услышанное
	\item \textbf{Рефлексивным}, который предполагает активное
	вмешательство в речь собеседника, оказание ему помощи в выражении своих мыслей и чувств, создание благоприятных условий для
	общения. Виды рефлексивного слушания:
	\begin{itemize}
		\renewcommand{\labelitemi}{--}
		\item \textbf{Выяснение} - это обращение к говорящему за уточнениями; оно помогает сделать сообщение более понятным слушающему.
		\item \textbf{Перефразирование} - это попытка сформулировать ту же
		мысль иначе.
		\item \textbf{Отражение чувств говорящего} понимание его установок и эмоционального состояния слушающим.
		\item \textbf{Резюмирование} высказывания помогает соединить фрагменты разговора в смысловое единство. Оно подытоживает основные идеи говорящего и весьма уместно в продолжительных беседах
	\end{itemize}
\end{itemize}
\subsection{Говорение}
Говорение — вид речевой деятельности, посредством которого
осуществляется устное общение. Речевое высказывание в процессе говорения создается на глазах слушателей. В связи с этим нужно отличать говоримую
речь (говорение) от озвученной письменной речи. Этим объясняются особенности говорения: оговорки, перебивы, паузы.

Типы:
\begin{itemize}
	\item Диалог
	\item Монолог
\end{itemize}


Механизмы привлечения внимания:
\begin{itemize}
	\item \textbf{Избыточность речи}. Она обусловлена стремлением автора активизировать внимание слушателей, заставить их задуматься над некоторыми вопросами, которые будут освещены в лекции, обратить внимание
	\item \textbf{Лаконизм речи} в процессе говорения, в ходе непосредственного общения с собеседниками автор высказывания может пропустить слово, если оно легко воспроизводится в конкретной ситуации общения; выразить какие-то компоненты содержания невербальными средствами общения (жестами), если используются
	таблицы, схемы, примеры, записанные на доске, и т. п.
\end{itemize}

В зависимости от сферы и ситуации общения и характера жан­
ра говоримая речь может быть неподготовленной (спонтанной),
частично подготовленной (частично-спонтанной) и подготов­
ленной.
\subsection{Чтение}
Чтение — вид речевой деятельности, в основе которого лежит восприятие и осмысление письменного текста. С помощью чтения человек реализует возможности так называемого
опосредованного общения.

Виды чтения:
\begin{itemize}
	\item \textbf{Ознакомительное чтение} предполагает беглое просматривание,
	фрагментарное, избирательное прочитывание текста, чтобы выявить в самом общем виде его характер.
	\item \textbf{Изучающее чтение} - глубокое, вдумчивое изучение какого-либо текста, требующего полного охвата его содержания, полного его осмысления. 
\end{itemize}

Приёмы чтения:
\begin{itemize}
	\item \textbf{Ознакомительное чтение}
	Оно направлено на общее понимание текста и начинается с анализа внешних данных книги:
	\begin{itemize}
		\item осмысление заголовка, эпиграфа;
		\item определение автора и знакомства с его работами;
		\item анализ выходных данных (дата, место издания);
		\item чтение аннотации;
		\item просмотр иллюстраций, схем и других нетекстовых элементов.
	\end{itemize}
	Цель — прогнозирование содержания до чтения. Важно уметь выделять ключевые слова, понимать структуру текста, обращать внимание на начальные фразы абзацев, указывающие на смысловые изменения.
	\item \textbf{Изучающее чтение}
	Оно предполагает глубокое понимание текста, его осмысление и запоминание. Используются те же приёмы, что и при ознакомительном чтении, но дополнительно:
	\begin{itemize}
		\item выявление скрытых вопросов;
		\item поиск ответов в тексте;
		\item выделение главного и сложного материала;
		\item представление содержания в виде схем, таблиц, рисунков для систематизации информации.
	\end{itemize}
	После чтения важно:
	\begin{itemize}
		\item сформулировать основную мысль;
		\item выделить ключевую информацию;
		\item переработать её для решения конкретных задач.
	\end{itemize}
	Эффективным считается не механическое запоминание, а умение переформулировать, делать выводы и обобщения. Это достигается с помощью механизма эквивалентных замен.
\end{itemize}

\subsection{Письмо}
Единицей общения является текст (высказывание).

\textbf{Признаки текста как единицы общения:}

\begin{itemize}
	\item \textbf{завершённость} — текст представляет собой законченное речевое произведение;
	\item \textbf{подчинённость единой теме} — вся информация организована вокруг одной темы;
	\item \textbf{раскрытие темы в соответствии с авторским замыслом};
	\item \textbf{адресованность} — текст направлен на конкретного адресата (друг, аудитория, читатель);
	\item \textbf{наличие целевой установки} — текст создаётся для достижения определённой цели;
	\item \textbf{включённость в речевую ситуацию} — высказывание существует во взаимодействии с другими высказываниями.
\end{itemize}

\textbf{Структура текста:}
\begin{itemize}
	\item предложения, воспринимаемые не сами по себе, а во взаимосвязи с другими
	\item сверхфразовые единства (фразовые единства, сложносинтаксические целые и др.)
	\item крупные смысловые блоки — абзацы, параграфы, части, главы
\end{itemize}

\textbf{Особенности текста:}
\begin{itemize}
	\item внутренняя связь между всеми единицами текста (лексическая, грамматическая, логическая)
	\item композиционное оформление, обусловленное типом высказывания (повествование, описание и т. д.)
	\item жанровая и стилистическая обработка, придающая тексту соответствующую форму.
\end{itemize}