\section{Говорение как вид речевой деятельности.}

\textbf{Говорение -- вид речевой деятельности, посредством которого осуществляется устное общение.}

Речевое высказывание в процессе говорения создается на глазах слушателей, оно творится в момент его произнесения, в момент его реализации. 
В связи с этим нужно отличать говоримую речь (говорение) от озвученной письменной речи (чтение стихов наизусть, зачитывание заранее написанного текста документов, распоряжений, приказов, доклада и т.~п.).
В процессе говорения мысль опережает слово, что порождает определенные трудности, связанные с оформлением высказывания, с подбором необходимых средств языка для передачи основной мысли создаваемого текста.
Этим объясняются и другие особенности говорения:
оговорки, пропуски слов, что свидетельствует о корректировке высказывания в процессе его создания на глазах слушателей; различного рода перебивы, срывы начатой конструкции, замена ее другой; наличие пауз.
Создавая устное высказывание в процессе говорения, автор автоматически использует речевые клише, типовые конструкции в сочетании с нестандартными оборотами.

В говорении проявляется сочетание противоположных явлений, связанных с использованием речевых средств: \textbf{лаконизма} (экономное говорение) и \textbf{избыточной речи}.
В процессе говорения, в ходе \textit{непосредственного общения }с собеседниками автор высказывания может пропустить слово,  выразить какие-то компоненты содержания невербальными средствами.
Наконец, если говорящий наблюдает за реакцией слушателей, он может не закончить начатое предложение (... вижу, что это понятно всем), свернуть высказывание (это ясно, идем дальше). Таким образом проявляется лаконизм речи в процессе говорения.
В говорении, в речи, которая создается в процессе непосредственного общения, проявляется личность автора, особенности его речевого поведения, специфика его характера, уровень интеллектуального развития, образования, культуры и т. п.

Говорение реализуется как в форме диалога (беседа, разговор, интервью), так и в форме монолога (лекция, ораторское выступление, ответ на экзамене, развернутые реплики в ходе дискуссии и т.~п.).
Но в любом случае диалогичность, стремление разговаривать со слушателями, вовлечение собеседников в процесс размышления является признаком и условием хорошего (настоящего) говорения, даже если оно реализуется в форме монолога