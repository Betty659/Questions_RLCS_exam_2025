\section{Риторический канон: история, теория, практика.}
	 \hspace{\parindent}\textbf{Риторический канон} (\textit{от греч. «правило», «предписание»}) - этапы подготовки речи. 

     Представляет собой 5 этапов подготовки:
     \begin{itemize}
         \item Инвенция (изобретение),
         \item Диспозиция (расположение),
         \item Элокуция (словесное выражение),
         \item Меморио (запоминание),
         \item Акция (словесное действие, произнесение речи).
     \end{itemize}

Все эти этапы следуют за осмыслением ритором ситуации общения и реализуются как в подготовленной, так и в спонтанной речи. 
Результатом подготовки речи на первых трех этапах становится текст, поэтому \textbf{изобретение, расположение и выражение называют основными этапами}. В современной терминологии эти три этапа составляют докоммуникативный этап подготовки речи.
    
Рассмотрим подробнее каждый этап подготовки речи.
Мы знаем, что главной задачей риторики является формирование коммуникативно-грамотной личности (языковой, речевой личности), которая умеет создавать тексты разных жанров, демонстрируя в них свою этическую (и эстетическую) сущность.

\textbf{Инвенция} (\textit{лат. inventio — «нахождение», «изобретение»})
В процессе подготовки этого этапа человек определяется с тем, что сказать. 
Процесс создания мыслей в риторике называется \textit{«изобретение»}.
Это проявляется \textit{«в способности найти уместное для данной аудитории содержание, развернуть доказательство, подобрать нужные аргументы, определить стиль общения с аудиторией» (Аннушкин)}.

Термин \textit{«изобретение» (inventio)} впервые встречается в античной риторике и означает нахождение, обретение мыслей и идей. Первая задача ритора заключалась в том, чтобы найти, что сказать.

Изобретение речи начинается с осмысления темы. 
Это значит, что ее необходимо разработать, то есть разделить на ряд подтем, которые в совокупности и составят тему. 
Но как это сделать? 
На этот вопрос \textbf{Цицерон} отвечает: \textit{«Найти и выбрать,что сказать, — великое дело: это — как бы душа в теле; но это забота, скорее, здравого смысла, чем красноречия, а в каком деле можно обойтись без здравого смысла? Конечно, оратор, в котором мы ищем совершенства, будет знать, откуда извлечь основания и доводы»}.

Как видим, процесс создания речи понимается и как технология, «работа здравого смысла», когда говорящий (пишущий) «будет знать, откуда извлечь основания и доводы», и в то же время это процесс творческий, требующий определенного настроя, которым сопровождается рождение идей, мыслей, слов. Античный образец предлагает готовую программу изобретения речей, в соответствии с которой можно разработать любую тему.

Первое теоретическое обоснование этапа инвенции и ответ на этот вопрос дает \textbf{Аристотель} в своем трактате \textbf{«Риторика»}. \textit{«Что касается способов убеждения — то их три вида: одни из них находятся в зависимости от характера говорящего, другие — от того или другого настроения слушателя, а третьи — от самой речи».}

Другими словами, изобретая содержание той или иной речи, говорящий должен думать сразу «в трех измерениях»:
\begin{enumerate}
    \item О том впечатлении, которое он сам производит на слушателей (это проблема нравственности говорящего и доверия к нему, проблема «образа говорящего», его честности и ответственности.
    \item О слушателях и их эмоциях («о страстях», по Аристотелю).
    \item  О доказательности самой речи («о доводах»).
\end{enumerate}

Итак, человек (ритор) при подготовке речи на этапе инвенции сначала должен обдумать систему способов убеждения, то есть определенную стратегию своего речевого поведения, которая позволила бы выступающему реализовать свою идею, основную мысль, интенцию, то есть цель выступления.
Именно этой интенции, которая должна быть реализована в стратегии речевого поведения («аргументативной стратегии») говорящий подчиняет отбор языковых и невербальных средств, элементов, на которые членится тема при ее понятийной разработке. Аргументы и мысли для своей темы, которые будут отражены в структурно — смысловых частях его будущего текста, говорящий выберет из набора традиционных «общих мест» (топосов). Эти аргументы и мысли, обличенные в слова, должны быть уместны и достаточны для понимания данным слушателем сущности предмета речи.

\textbf{Диспозиция} \textit{(лат. dispositio — «расположение»)}

 После того как вы отобрали материал для своего будущего выступления, определились с тезисом, фактами, аргументами, цитатами и т. д. перед вами стоит задача отобрать и разместить весь этот материал так, чтобы и вам его было удобно произносить, и слушателям было удобно его воспринимать на слух.
 
Значит, необходимо этот материал расположить, выстроить определенным образом — в наиболее оптимальном для понимания слушателями порядке, в естественной для восприятия аудиторией последовательности движения вашей мысли, как единственно возможное для данной аудитории и данного оратора содержание, \textit{«ибо что пользы есть в великом множестве разных идей, ежели они не расположены надлежащим образом?» (М. В. Ломоносов)}.

Таким образом, диспозиция как структура расположения элементов речи, полученных в результате инвенции, представляет собой учение:
\begin{itemize}
    \item  о соразмерном членении и организации материала (соотношение его отдельных частей, отношение каждой части ко всему выступлению в целом);
    \item о внутренней (план) и внешней (композиция) структуре;
    \item об универсальной композиционной схеме и о частных композиционных схемах;
    \item о функционально-смысловых типах речи (повествование, описание, рассуждение);
    \item о линейной синтаксической группировке элементов содержания (микротем) с целью создания определенного ритма речи.
\end{itemize}

\textbf{Общие правила расположения мыслей, по М. М. Сперанскому}, таковы:
\begin{enumerate}
    \item Все мысли в слове должны быть связаны между собой так, чтобы одна мысль содержала в себе, так сказать, смысл другой.
    \item Все мысли должны быть подчинены одной главной. Во всяком сочинении есть известная царствующая мысль: к сей-то мысли должно все относиться. Каждое понятие, каждое слово, каждая буква должны идти к концу: иначе они будут введены без причины, они будут излишни.
\end{enumerate}

Диспозиция как результат обдумывания предстоящей речи — в общем виде — содержит в себе: 
\begin{itemize}
    \item введение в тему («ангажирование сюжета»);
    \item развертывание, развитие темы (повествование, описание, доказательство);
    \item подытоживание темы.
\end{itemize}

Перечислим виды структурных схем: 
\begin{itemize}
    \item трехчастная:
        \begin{enumerate}
            \item Введение.
            \item Основная (главная) часть.
            \item Заключение.
        \end{enumerate}
    \item естественная композиция (более подробная, нежели трехчастная):
        \begin{enumerate}
            \item Введение.
            \item Основная часть:
                \begin{enumerate}
                    \item  Изложение (наррация - рассказ о действиях, поступках событиях, нужна для убедительности).
                    \item Аргументация:
                        \begin{enumerate}
                            \item Позитивное доказательство.
                            \item Негативное доказательство (опровержение точки зрения 				противника).
                        \end{enumerate}
                \end{enumerate}
        \item  Заключение:

            ((a) Резюме.)
        \end{enumerate}
    \item семичастная:
        \begin{enumerate}
            \item вступление (1 — обращение; 2 — называние, именование темы);
            \item предложение (теорема);
            \item повествование (наррация, история вопроса);
            \item описание (рассказ о положении дел на момент речи);
            \item рассуждение (1 — доказательство (выдвижение аргументов) и 2 — опровержение (рассмотрение контрдоводов));
            \item воззвание (обращение к чувствам слушателей);
            \item заключение (концовка или резюме).
        \end{enumerate}
\end{itemize}

Сказанное позволяет сделать значимые выводы. 
Из приведенных схем видно, как постепенно возникают и приобретают определенность обобщенные, типовые схемы представления содержания (информации) внутри речи: \textbf{описание, повествование, рассуждение}. Получается, что все речевые произведения по своим функционально-смысловым характеристикам и особенностям синтаксических связей имеют определенное обобщенное значение.

Каким же образом происходит выбор говорящим / пишущим типовой схемы речи, функционально-смыслового типа речи?  Риторика утверждает, что этот выбор зависит от многих факторов, в том числе от отношения оратора к предмету речи и от тех целей, которые автор речи ставит перед собой в данном выступлении, в конкретном тексте. 

Если задачей оратора является \textit{потребность поделиться впечатлениями со слушателями}, вызвать определенные эмоции у аудитории, описать кого-либо, что-либо, то он воспользуется типом речи \textbf{описание}. 

Если говорящий желает \textit{рассказать о каком-то событии}, происшествии, перечислить
какие-либо действия предмета речи, а также намерен побудить слушателей к какой-нибудь деятельности, то он обратится к \textbf{повествованию}. 

При необходимости \textit{убедить или разубедить адресатов в какой-либо точке зрения}, адресант построит свою речь в форме \textbf{рассуждения}. При этом оратор может выразить \textit{свое отношение к тому, о чем он говорит}, прокомментировать, оценить это, тогда ему следует прибегнуть к \textbf{оценке}.

\textbf{Элокуция} \textit{(лат. elocutio — «словесное выражение»)}

Этот этап включает в себя словесное оформление мысли, «украшение» её словами. Этот этап риторического канона завершает «путь от мысли к слову, путь постепенного нахождения \textit{«пристойных»}, по утверждению Ломоносова, слов и выражений и «украшение» всего высказывания.

Только на этом этапе мысль обретает словесные одежды, созданный каркас воплощается в целостную конструкцию — архитектурно завершенное здание.

Еще античными риторами были разработаны \textbf{критерии красноречия: ясность, правильность, уместность употребления тех или иных приемов ораторского искусства, словесные украшения} («цветы красноречия»), которые допустимо использовать.

Средства художественной выразительности допустимо и необходимо использовать не только в художественных текстах, но и в научных и публицистических, дабы подчеркнуть значимость той или иной информации, доступно и доходчиво выразить мысль, или же усилить воздействие мысли, придать ей эмоциональную окраску.

Такими средствами являются \textbf{тропы} — это слова и обороты речи, которые усиливают ее выразительность.

\textbf{Примеры:}
\begin{itemize}
    \item \textbf{метафора} — перенос свойства одного предмета на другой по принципу их сходства или контрасту («Туча легко вскинулась на дыбы» (Б. Л. Пастернак)); 
    \item \textbf{метонимия} — замена имени (« Я три тарелки съел» (И. А. Крылов));
    \item \textbf{синекдоха} — разновидность метонимии — целое или общее выражается через свою часть, множественное число употребляется вместо единственного и наоборот («Москва заходит в Латинскую Америку с дамы», загадочным этот заголовок можно назвать потому, что содержит в себе три переименования, которые обозначают: президент России Дмитрий Медведев провел переговоры с президентом Аргентины Кристиной Кишнер);
    \item \textbf{ирония} — придание словам смысла, противоположного их буквальному значению.
\end{itemize}

Помимо тропов существуют \textbf{риторические фигуры}. К ним относятся: \textbf{антитеза} (противопоставление), \textbf{повторы} (не засоряющие речь, а риторические фигуры!), \textbf{анафора} (единоначатие), \textbf{эпифора} (единоокончание), \textbf{параллелизм} (разновидность риторического повтора, на основе использования определенной синтаксической конструкции, в которой тождественно располагаются элементы речи в смежных частях предложения, следующих друг за другом), \textbf{симплока} (фигура речи, сочетающая анафору и эпифору). 

\textbf{Риторические вопрос, восклицание, обращение относятся к средствам диагонализации} — средствам, которые устанавливают мысленное взаимодействие оратора и адресата.

Теперь рассмотрим основные задачи ритора на последних двух этапах — запоминании и самом выступлении.

\textbf{Меморио} \textit{(лат. memorio — запоминание )}

Основные задачи ритора:
\begin{itemize}
    \item проверить удобство текста для устного произнесения;
    \item запомнить начало, концовку и формулировки главных положений выступления, основные цифры и факты;
    \item  отрепетировать текст в плане синхронизации подготовленного текста с электронной презентацией.
\end{itemize}

\textbf{Акция} \textit{(лат. actio — действие, разрешение)}

Основные задачи ритора:
\begin{itemize}
    \item устранить озвучивание письменного текста;
    \item сохранять высокий уровень устности речи; 
    \item уместно применять невербальные средства речи;
    \item установить и поддерживать акустический и визуальный контакт с аудиторией; 
    \item постоянно использовать средства управления вниманием слушателей;
    \item следить за непрерывностью получения обратной связи от аудитории; 
    \item синхронизировать речь и слайды презентации;
    \item соблюдать приоритет высказывания ритора по отношению к электронной презентации (презентация должна играть только вспомогательную роль). 
\end{itemize}
Риторический канон — это универсальный алгоритм создания убедительной речи. Освоив все пять этапов, можно эффективно доносить мысли до любой аудитории. 
