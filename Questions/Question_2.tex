\section{Язык как система. Уровни языковой системы, их взаимосвязь.}

Язык представляет собой систему (от греч. $\sigma \upsilon \sigma \tau \eta \mu \alpha$ --- целое, составленное из частей, соединение) знаков, за которыми закреплено соответствующее их звуковому облику содержание.
Поясним, какое понимание вкладывается в ключевые слова этого положения.

Язык --- система \textbf{знаков.}
Это самые главные слова, характеризующие язык.
Язык становится языком только тогда, когда за каждым звуком, словом или предложением этого языка стоит то или иное значение, которое может придать этому знаку определенный смысл.
Например, звуки [да] в русском языке имеют значение --- они могут при соответствующей интонации выразить согласие.

Язык --- \textbf{система} знаков, то есть эти единицы языка не случайны, они взаимосвязаны, они образуют единство, которое функционирует только целиком.
При этом каждая единица этой системы представляет собой частицу целого.
Система любого национального языка состоит из единиц, объединенных на соответствующих уровнях: фонемы (звуки речи) образуют фонемный уровень, морфемы (части слова) --- морфемный, слова --- лексический, словосочетания и предложения --- синтаксический.
В свою очередь, каждый уровень включает соответствующие единицы языка: предложения состоят из слов, слова --- из морфем, а морфемы --- из фонем.
Между всеми этими и многими другими единицами языка возникают сложные взаимоотношения, которые и определяют единство и целостность всей языковой системы, предназначенной для выполнения различных многообразных функций языка.

При этом каждая единица языка обладает определенным и всеми признаваемым значением, которое позволяет использовать данный язык в качестве основного средства отправления и приема информации, передачи и восприятия социального опыта, сохранения национальной культуры, которая неотделима от языка.

Роль языка в жизни каждого общества огромна, поскольку возникновение и существование человека и его языка неразрывно связаны друг с другом.
«Язык предназначен для того, чтобы служить орудием общения людей, и устроен так, чтобы быть естественно усваиваемым и адекватным средством обмена информацией и ее накопления.
Его структура подчинена задачам коммуникации, которая состоит в передаче и приеме мыслей об объектах действиельности» (Русский язык. Энциклопедия).

Человеческий язык отличается от так называемого языка животных, представляющего собой набор сигналов-реакций на ситуацию, прежде всего тем, что с помощью языка люди передают друг другу не только конкретную, но и отвлеченную информацию, которая является плодом мышления, а также тем что основные правила употребления языка не только ощущаются носителями этого языка, но и осознанно соблюдаются.
Таким образом, человека от других живых существ отличает не только то, что он умеет мыслить (\textit{homo sapiens}) и что он --- человек-созидатель (\textit{homo faber}), но и то, что он человек говорящий (\textit{homo eloquens}) и человек общающийся (\textit{homo communicans}).

Разум человека и его потребности в языках, способных наиболее адекватно выразить смысл во всех областях человеческой жизни, привели к тому, что человек пользуется как национальными \textbf{языками} --- \textbf{естественными,} существующими с незапамятных времён: русским, английским, японским и др., так и им самим созданными новыми --- \textbf{искусственными.}
Искусственные языки сейчас весьма разнообразны, они обслуживают различные сферы жизни, являются международными, поскольку не ограничены национальными рамками.
К искусственным языкам относятся прежде всего созданные на базе естественных национальных языков международные: эсперанто, волапюк и др.
Кроме того, искусственные языки --- это символические языки науки: языки математики, логики, химии и др.
Искусственными языками являются и языки человеко-машинной коммуникации --- программирования, управления базами данных и т.п.: фортран, алгол-60 и пр.

\subsection*{Функции естественного национального языка.}

Главное предназначение языка --- служить основным средством обмена информацией (то есть выполнять \textbf{коммуникативную функцию}).
Иначе говоря --- для общения.
Мы говорим друг с другом на русском языке, передавая и воспринимая таким образом самую разнообразную информацию.

Вторая важнейшая функиия --- быть основной формой отражения окружающей человека действительности и самого себя, а также средством получения нового знания о действительности (то есть выполнять \textbf{познавательную,} или \textbf{когнитивную, функцию}).

Таким образом, любой естественный человеческий язык предназначен прежде всего для общения и познания действительности.

К основным функциям языка относятся также \textbf{эмоциональная} (быть одним из средств выражения чувств и эмоций) и \textbf{метаязыковая} (быть средством исследования о описания языка).
Эмоциональная функция языка очень важна для человека, поскольку помогает ему выразить свой внутренний мир, свои впечатления, ощущения, оценки и~т.~п. наиболее адекватно, тем более, что большинство высказываний на том или ином языке содержат не только логическую, но и эмоциональную информацию.
Метаязыковая функция в повседневной жизни играет меньшую роль, но эта книга и другие письменные и устные тексты о языке выполняют в немалой степени именно эту функцию.

В составе основных функций выделяются и другие.
Так, осуществлению коммуникативной функции способствуют \textbf{фактическая} (контактоустанавливающая), \textbf{усвоения информации, воздействия,} а также \textbf{кумулятивная функция} (создания, хранения и передачи информации).
Кроме того, у языка есть и \textbf{эстетическая функция,} которая предполагает, что сама речь и ее фрагменты могут восприниматься как прекрасное или безобразное, то есть как эстетический объект, и \textbf{аксиологическая} (функция оценки), и др.

И все эти функции объединяет то, что язык предназначен и существует не для отдельного индивида, а для определенного общества, в котором этот язык выступает в роли общего кода, с помощью которого люди и способны понимать друг друга.

Однако язык выполняет эти функции только тогда, когда используется в процессе речи для создания высказывания.
Таким образом, язык предназначен для выполнения этих функций, но сам по себе язык, без усилий говорящего на нем или пишущего, эту роль, как и другие свои функции, выполнить не может.
