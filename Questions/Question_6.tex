\section{Среднелитературная речевая культура.}

\begin{itemize}
    \item высшее или среднее, среднее специальное образование носителя языка;
    \item преимущественно рефлексивный тип интеллекта;
    \item некатегоричность в оценках;
    \item неудовлетворенность своим интеллектуальным багажом, наличие потребности в расширении своих знаний и их проверке;
    \item выполнение работы, постоянно требующей определенных интеллектуальных усилий;
    \item соблюдение основных этических норм;
    \item соблюдение основных норм речевого этикета;
    \item соблюдение основных норм литературной речи, усвоенных в школе;
    \item владение основными стилями устной речи;
    \item примерно одинаковое владение культурой устной и письменной речи;
    \item способность достаточно легко менять стиль и жанр речи с изменением коммуникативной ситуации;
    \item способность контролировать и изменять свою речь в ее процессе (тематический и стилистический контроль);
    \item отсутствие общей самоуверенности;
    \item отсутствие языковой самоуверенности;
    \item «неперенос» того, что типично для устной речи, в письменную речь и, напротив, того, что свойственно письменной речи, в устную;
    \item владение основами связной устной монологической речи, способность без подготовки или с минимальной подготовкой выступить с небольшим устным монологом на известную тему;
    \item привычка обращаться к словарям или специалистам-филологам для уточнения значений слов, привычка спрашивать других о том, «как правильно сказать»;
    \item фиксация языковых нововведений в речи на радио или телевидении, в СМИ, комментирование их в семье или профессиональной среде,
    \item богатство как активного, так и пассивного словаря;
    \item способность использовать синонимы в своей речи;
    \item знание основных изучавшихся в школе прецедентных текстов художественной литературы, способность использовать в качестве цитат в непосредственном общении некоторые(известные из школьной программы) строки;
    \item владение основными нормами устного речевого этикета;
    \item владение эпистолярным жанром (регулярно пишет письма);
    \item способность самостоятельно готовить необходимые письменные документы с опорой на собственные языковые знания;
    \item способность к «языковой игре», получение от нее удовольствия;
    \item умение использовать сниженную лексику и фразеологию экспрессивных, художественно-изобразительных целях;
    \item понимание речевого юмора;
    \item понимание подтекста в шутке, анекдоте, пословице, а также в художественном тексте – вербальном, визуальном, креолизированном;
    \item получение удовольствия от восприятия сложных текстов и теоретических дискуссий;
    \item нелюбовь к примитивным диалогам в вербально и визуально воспринимаемых текстах;
    \item способность оценить как форму, так и содержание текста;
    \item интерес к новым словами и выражениям, иностранным словам;
    \item в речи присутствует тематический и стилистический самоконтроль;
    \item экспрессивность речи создается интонацией, образностью, синонимическими средствами языка, юмором, использованием прецедентных текстов, уместных цитат, а не сниженной лексикой;
    \item понимание того, что важно не только, ЧТО сказать, а и КАК сказать;
    \item неиспользование жаргонной и ненормативной лексики, редкие случаи использования имеют целью создание шутливого колорита и соответствующим образом оформлены;
    \item отсутствие в речи аргументов типа«все так говорят» или «по радио,
    \item телевидению так говорили, я слышал»;
    \item изменение тональности общения в зависимости от коммуникативной
    \item ситуации;
    \item использование преимущественной формы \textit{вы}-общения, соблюдение норм употребления ТЫ и ВЫ;
    \item речь не содержит общеупотребительных штампов;
    \item фиксирование ошибок в речи других --- в устной речи, письменных и медийных текстах, в рекламе.
\end{itemize}
