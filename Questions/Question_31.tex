\chapter{\textbf{Информирующая речь}}
\chapter{\textbf{Что представляет собой информирующая речь? Каковы ее особенности?
\textbf{Информирующая речь} — один из видов устных высказываний, ее коммуникативная цель — дать новые сведения о том или
ином предмете, пробудить интерес к предмету речи. К особенностями информирующей речи можно отнести: удовлетворение потребности слушателей в получении исчерпывающей информации по данному вопросу; пробуждение любознательности; актуальность для данной аудитории; демонстрация нерешенных проблем; наличие и анализ конкретных фактов; сопоставление старого и нового, подчеркивание новизны. В структурном плане основная часть информирующей речи состоит, как правило, из 2—3 разделов. Тематическая неоднородность не характерна для одной информирующей речи, если только это не определяется особенностями конкретного жанра (выпуск новостей). Обилие проблем свидетельствует о непродуманное, поверхностном взгляде говорящего. Информирующая не значит неглубокая по содержанию. Речь любого говорящего человека содержит и передает информацию. Другое дело, что эта информация может быть разной: объективная научная информация — логическая, информация о чувствах — эмоциональная информация. Включите радио или телевизор: через определенные промежутки времени можно услышать или увидеть заставку — \textit{«Информационный выпуск»}. Это особый речевой жанр, цель которого — передать по возможности объективную информацию о политических событиях, событиях культурной и спортивной жизни и т. д. для возможно более полного усвоения слушателями. Посмотрите, как строятся такие информационные выпуски: они содержат комментарий диктора, высказывания официальных лиц, высказывания других средств массовой информации, в том числе зарубежных агентств, включения с места событий в режиме \textit{«онлайн»}. Такая структура придает излагаемой информации официальность и достоверность, позволяет воздействовать на чувства слушателя и зрителя.
Таким образом, информирующие жанры — это всегда сплав объективного (основной содержательный элемент) и субъективного, эмоционального (дополнительный содержательный элемент, вместе с тем очень важный, так как не только содержит блок, подтверждающий официальную версию, но и воздействует на чувства слушателя или зрителя). Следовательно, несмотря на то что основная коммуникативная \textbf{задача информирующей речи — передать в устной речи информацию для аудитории}, данный вид содержит также блок (гораздо меньший по объему и менее важный), включающий эмоционально окрашенную воздействующую информацию, подтверждающую или опровергающую объективную информацию. Текст информирующей речи написан заранее, диктор воспроизводит, озвучивает написанный текст. Это связано с тем, что материал официальный и никакие оговорки, \textit{«от себя»} вставленные слова не должны изменить единой избранной версии. Такие оговорки и свое мнение допускаются в репортаже с места событий, где передается эмоциональная субъективная информация специального корреспондента, неофициальные комментарии политиков, известных людей, просто случайных прохожих, свидетелей. 
\textbf{В чем особенности доклада одного из жанров информирующей речи?}
Существуют жанры научно-информативного типа, с которыми приходится иметь дело всем студентам во время обучения, довольно часто — школьникам и даже иногда взрослым. Это доклады на научную или научно-популярную тему. Имея много общего с остальными информирующими жанрами вообще, и с жанром информационных выпусков в частности, научно-информативные жанры обладают и своими особенностями. \textbf{Коммуникативная задача сохраняется — передать в устной форме информацию для максимально полного усвоения слушателями.} Сохраняется и публичный характер высказывания: аудиторией докладчика является минимум студенческая группа. Для жанра доклада характерна предварительная подготовка, хотя более подробно о подготовке и воспроизведении подготовленного речь пойдет дальше. Наконец, еще одной особенностью научно-информативных жанров, к которым относится доклад, является учет того обстоятельства, что слушатели могут фиксировать необходимую информацию. Отсюда — порционная подача этой информации, четкие формулировки, произнесение особо важных моментов \textit{«под запись»}. Есть и другие особенности, которые будут сформулированы ниже.
\textbf{Доклад }— это сообщение о постановке проблемы, возможных путях ее решения, о ходе исследования, его результатах. Различают научный доклад и учебный доклад. Научный доклад содержит объективно новые сведения. В учебном докладе эта новизна — понятие субъективное. Для учебного доклада может быть отобрана информация, субъективно важная для готовящего доклад студента или ученика. Его могут интересовать новые факты, новые подходы, своеобразная их интерпретация, наконец, возможность самостоятельно сделать свои выводы, сформулировать свою позицию.
\textbf{Каковы наиболее частые ошибки при подготовке доклада и выступления с ним?}
\textit{ «Скачивать»} рефераты и доклады можно только в том случае, если вы совсем не заинтересованы в собственном интеллектуальном и речевом росте и росте окружающих, если заранее запланировать доклад как работу \textit{«для галочки»}.
Настоящий доклад требует определенных усилий. Самые частые ошибки связаны, как правило, со следующими моментами:
— \textbf{в плане содержания}: а) доклад представляет собой выписки из одной или нескольких научных работ, докладчик не пропустил этот материал через свое сознание, не переформулировал текст, не проиллюстрировал примерами, более близкими аудитории; б) язык слишком сложный для восприятия на слух — это оттого, что письменная речь имеет свои особенности, поэтому просто озвучить написанное другим человеком, пусть даже на бумаге все выглядит гладко и хорошо, нецелесообразно: нужного эффекта не будет; в) доклад не структурирован, а ведь \textit{«публичное выступление — это путешествие с определенной целью, и маршрут должен быть нанесен на карту. Тот, кто не знает, куда он идет, обычно приходит неизвестно куда»} (Д. Карнеги); г) не высказаны собственные мысли по поводу изложенного, а ведь это наиболее интересная часть доклада: согласен ли говорящий с автором книг, по которым готовился, какие моменты представляются спорными, какие — просто неубедительными, в чем видит решение проблемы сам говорящий; - не используются никакие средства диалогизации, привлечения внимания: ни риторические вопросы, ни отвлечения с приведением собственных пример, ни непосредственные обращения к аудитории;
-  \textbf{в плане воспроизведения}: а) говорящий не может установить контакт с аудиторией, он уткнулся в свои записи, поднимает глаза к потолку или, что встречается достаточно часто, обращается взглядом к преподавателю; б) руки мешают: докладчик скрещивает их на груди, крутит предметы или стоит по стойке «смирно»; в) недостатки дикции: голос излишне тихий, робкий, маловыразительный в плане интонационного оформления, темп речи замедленный или, напротив, слишком быстрый; г) текст читается без всяких отвлечений. Любой преподаватель подтвердит, что эти недочеты являются наиболее частыми при выступлении с подготовленным докладом в аудитории.
 \textbf{Как построить работу над докладом?}
Обратившись к теории речевой деятельности, можно попробовать осуществить подготовку доклада, соотнося ее со следующими фазами:
-  \textbf{1 -й этап} работы побудительно-мотивационный окончательный выбор темы: подумайте, что вы лично знаете, приставляете по данной теме, каким жизненным опытом располагаете,затем возможен подбор литературы, чтение этой литературы.
 \textbf{2-й этап} — ориентировка - мысленный отбор наиболее интересного, отбрасывание ненужного, не отвечающего формулироаке, продумывание структуры доклада: какова будет главная мысль (тезис), которую необходимо доказать, какие аргументы и примерыв из текста могут служить доказатествами в данном случае; есть ли свои, жизненные, примеры; выработка собственной позиции согласия или несогласия с автором  книги, попытка найти другое, отличное мнение, определить свое к нему отношение. Здесь же необходимо продумать места пауз (особенно если учесть, что слушатели могут записывать отдельные положения доклада — им надо дать такую возможность), повышения или понижения голоса, выделения голосом наиболее важных моментов. Особое внимание целесообразно уделить взгляду и жестам. 
 \textbf{3-й этап} — исполнительский — запись продуманного. Конечно, можно и необходимо пользоваться материалом первоисточника, но целесообразно переформулировать его, излагая мысли более доходчиво, используя средства повышения внимания и средства диалогизации: обращения к аудитории, риторические вопросы, активизация общих воспоминаний и т. п. Существуют различные точки зрения, нужно ли записывать свое будущее выступление или достаточно ознакомиться с материалами, а потом положиться на импровизацию. Но, во-первых, давно известно, что самый лучший экспромт это тот, который заранее хорошо подготовлен, во-вторых, практика предшествующих поколений показывает, что в прошлом речи писались, особенно это относится к Древней Руси. Все \textit{«Поучения»} и \textit{«Слова»} создавались как литературные произведения и распространялись в списках. В-третьих, опыт известных риторов свидетельствует: \textit{«Мы не будем повторять старого спора: писать или не писать речи. Знайте, читатель, что, не исписав несколько сажен или аршин бумаги, вы не скажете сильной речи по сложному делу. Если только вы не гений, примите это за аксиому и готовьтесь к речи с пером в руке»} (П. С. Пороховщиков (Сергеич). Написанный текст можно проверить, отточить формулировки, что позволит избежать оговорок, повторов и, напротив, избавит от пропусков важного. \textbf{4-й этап} — произнесение доклада вслух, репетиция последующего выступления. Вот здесь желательно уподобиться Демосфену и попробовать не только произнести текст, хотя это самое главное, но и \textbf{проинтонировать} его, проверить, как получается использование жестов, как устанавливается контакт с воображаемой аудиторией при помощи взгляда. Эта фаза для того и существует , чтобы внести необходимые изменения как в текст, так и в поведение. На слух легче определить, какая фраза слишком затянута какая — неточно сформулирована. Единственное, что исключается, — чтение написанного. Озвученная письменная речь - далеко не лучший доклад: его содержание в такой форме плохо воспринимается аудиторией, а поскольку глаза заняты чтением и не могут оторваться от текста, то зрительный контакт со  слушателями исключен. К. тому же зачитывание у многих слушателей ассоциируется с плохой подготовкой и некомпетентностью. При этом нельзя не отметить, что есть ситуации, когда допустимо и возможно только чтение. Речь идет, например, о защите дипломных проектов, когда неточность формулировок просто недопустима. Кроме того, необходимо уложиться в строго отведенное время, а это невозможно без записанного текста и предварительной репетиции. Конечно, можно выучить свое выступление или доклад наизусть, но это вряд ли необходимо. Кроме того существует опасность, что в том случае, если говорящий наизусть собьется, он не сразу сможет собраться с мыслями и продолжить.

\textbf{Какие средства наглядности целесообразно использовать, выступая с докладом?}
Наглядность в речи может быть двух типов: а)\textbf{образная речь}; б)\textbf{использование средств наглядности}: плакатов, таблиц, графиков и т.д. Использование образной наглядности не только влияет на качество запоминания, но и упрощает мысль, идею для восприятия и тем самым делает ее более понятной и убедительной. Использование средств наглядности также требует, по П.Соперу и Ф.Снеллу, выполнения некоторых правил: \textbf{во-первых}, средства наглядности можно и нужно использовать лишь тогда, когда они действительно необходимы для пояснения или возбуждения (поддержания) интереса к излагаемому материалу; \textbf{во-вторых}, заранее ничего из средств наглядности вывешивать или открывать не следует, это делается лишь в нужный момент; \textbf{в-третьих}, не использовать таблицы и графики, если они плохо будут видны аудитории; \textbf{в-четвертых}, статистическим таблицам правильнее придавать вид диаграмм, желательно в форме разноцветных прямоугольников, отражающих размеры, тенденции и т. д.; \textbf{в-пятых}, обязательно увязывать слова с изображением на таблицах и графиках, обращаясь при этом не к пособиям, а к слушателям; сделать небольшую паузу, дать возможность разглядеть таблицу; \textbf{в-шестых}, не раздавать слушателям никаких пособий, так как это снижает уровень внимания; как только изображения на доске или таблицы стали не нужны, их необходимо убрать; и, наконец, если демонстрируется предмет, его надо держать в руке на уровне плеч или чуть выше. 

\textbf{Как можно использовать диалогичность при подготовке и выступлении с докладом перед аудиторией?}
Одной из важнейших составляющих любого доклада является диалогичность — сотрудничество оратора и аудитории. Диалогичность предполагает наличие доброжелательного, понимающего слушателя. Но диалогичность - понятие довольно абстрактное. \textbf{Реализуется она в речи с помощью особых приемов}: обращение к аудитории в начале и по ходу доклада; апелляция к авторитету слушателей; использование риторических вопросов; введение афоризмов, пословиц и поговорок, предполагающих однозначную реакцию слушателей; искреннее выражение оценки того или иного факта ожидание ответного сопереживания от слушателей; драматизация изложения, сопоставление всех «за» и «против» при решении поставленного вопроса и т. д. Самое правильное — когда докладчик представляет аудиторию не в виде емкости, куда можно «слить» информацию, а в качестве равноправного партнера с предсказуемой реакцией, на которую можно опираться. Поэтому все в выступающем :доброжелательный тон, интонация, тембр голоса, темп речи, паузы — должно выражать доброжелательное отношение к аудитории. Это, в свою очередь, действительно приводит к возникновению взаимопонимания между оратором и слушателями, что в еще в большей степени способствует решению поставленных докладчиком проблем.
