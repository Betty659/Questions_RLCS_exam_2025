\section{Культура речи и базовая культура личности. Формирование языковой (речевой) личности.}

\subsection*{Что такое культура}

Само слово культура зародилось в Древнем Риме. «Как известно, слово "культура" имеет в качестве исходного латинское "cultura", что означало и "обрабатывать" (землю), и "совершенствовать", и "почитать".
В позднейшем употреблении слова "культура" сохранялись эти оттенки, но любопытно, что первоначально "культура" означала изменение природы в интересах человека, точнее - возделывание земли.
И параллельно возникает метафора, употребляемая Цицероном, --- "культура (совершенствование) души", "духовная культура"», --- пишет А.~А.~Брудный.

\subsection*{Культура русской речи}
 

Культура речи включает в себя умение четко и ясно выражать свои мысли, говорить грамотно, привлекать внимание аудитории не только содержанием своего выступления, но и эмоциональным воздействием на слушателей.
Владение культурой речи --- своеобразная характеристика профессиональной пригодности людей самых различных профессий, которые по роду деятельности организуют и направляют работу, ведут деловые переговоры, воспитывают, оказывают разного рода услуги.

\textbf{Культура речи предполагает:}

\begin{itemize}
    \item соблюдение правил речевого общения;
    \item владение нормами литературного языка в его устной и письменной формах;
    \item умение выбрать и организовать языковые средства, которые в конкретной ситуации общения способствуют достижению определенных коммуникативных целей.
\end{itemize}

\textbf{Таким образом, культура речи содержит три аспекта: нормативный, коммуникативный и этический.}

Важнейшим является \textbf{нормативный аспект.} Он отражает правильность речи, то есть соблюдение норм литературного языка.
Языковая норма --- центральное понятие языковой культуры.
\textit{Умение правильно говорить --- еще не заслуга, а неумение --- уже позор, потому что правильная речь не столько достоинство хорошего автора, сколько свойство каждого гражданина,} --- утверждал знаменитый римский оратор Цицерон.

Культура речи не может быть сведена к перечню запретов.
Необходимо иметь навыки отбора и употребления языковых средств в соответствии с коммуникативными задачами.
Это основа коммуникативного аспекта культуры речи.

Известный лингвист Г. О. Винокур писал: \textit{Для каждой цели свои средства, таков должен быть лозунг лингвистически культурного общения.
Носители языка должны владеть разными функциональными стилями, чтобы осуществлять оптимальный выбор языковых средств.}

Этический аспект предписывает знание этических норм речевого поведения и предполагает уместное использование речевых формул приветствия, просьбы, вопроса, благодарности, извинения, прощания и~т.~п.

\textbf{Нарушение этики} общения приводит к \textbf{коммуникативным неудачам}, типа: \textit{У меня (есть) вопрос} или \textit{Есть вопрос} вместо \textit{Разрешите / позвольте задать Вам вопрос} или \textit{Скажите, пожалуйста}.
Подобные коммуникативные неудачи связаны с нарушением рамок общения, которые зависят от характера отношений между собеседниками (официальные, неофициальные, дружеские, интимные). Этический аспект культуры речи накладывает \textbf{строгий запрет на сквернословие, повышенный тон, деликатные (табуированные) темы.}

Для современной просторечной речевой культуры характерно неразличение сферы \textbf{ты- и Вы- обращения.}

Форма \textbf{Вы} была заимствована из западноевропейских языков в XVII-XVIII вв.
Одна из гипотез ее появления такова.
В эпоху распада Римской Империи, когда два императора занимали престолы в Риме и Константинополе, при обращении к каждому из них использовали форму множественного числа, дабы не обидеть другого. Исконной для русского языка является форма \textbf{ты,} о чем свидетельствуют молитвенные обращения к Богу.

В некоторых сообществах, социальных группах, особенно \textbf{в среде интеллигенции}, даже при хорошем знакомстве может сохраняться обращение на \textbf{Вы}. В \textbf{просторечной среде отдается предпочтение обращению на ты} как демонстрации простоты, равенства, доверительности.

Современные тенденции, особенно в средствах массовой информации (СМИ), игнорировать форму Вы свидетельствуют о серьезном нарушении норм русской речевой культуры и о неудачных попытках слепого подражания Западу. Нередко журналист,
гордясь своим личным знакомством с солидным политиком, ученым, бизнесменом, называет его на ты, в то время как хорошо воспитанный человек будет испытывать смущение, обращаясь к даме или к господину возраста его родителей на ты.

\textbf{К этическому аспекту культуры речи} относится и правильное обращение к собеседнику.
Наука о личных именах носит название \textbf{sантропонимии}.
Выбор личного имени определяется национально-ментальным стереотипом.
Личное имя имеет несколько
понятийных опор: одушевленность, пол, возраст.

Русские личные имена живут активной социальной жизнью и составляют значительный пласт лингвокультурологически значимой лексики, то есть таких слов, в которых отражается культурная история народа.
Русский классик А.~И.~Куприн писал, что \textit{язык --- это история народа, путь цивилизации, культуры, поэтому изучение и сбережение русского языка является не праздным занятием, от нечего делать, но насущной необходимостью.}

Большинство современных личных имен иноязычного происхождения.
В Древней Руси бытовали собственно русские имена:
\textit{Ветер, Любава, Забава, Добрыня}. Очень часто ребенку после крещения давали греческое имя (например,\textit{ Александр, Ксения}),
но в миру называли именем-оберегом (\textit{Волк, Сила}).

В 20-е годы прошлого века нашу страну захлестнула мода на аббревиатуры, которая отразилась и в личных именах.
Появились дети с именами Ким (\textit{Коммунистический Интернационал Молодежи}), Рэм (\textit{Революция, Энгельс, Маркс}), Велиор (\textit{Великая Октябрьская Революция}), Сталина, Даздраперма (\textit{Да заравствует Первое Мая!}).

После смерти В.И. Ленина мальчиков нередко называли именем Вилен.
В 40-е годы популярными именами были Мэлс (\textit{Маркс, Энгельс, Ленин, Сталин}) и Польза (\textit{Помни ленинские заветы}). В 50-е годы появились имена Мират (\textit{Мирный атом}), Нинель (\textit{Ленин --- в обратном порядке}).

В русской коммуникативной культуре \textbf{обращение по имени и отчеству} предпочтительно, так как отчество составляет важную национальную особенность русской речевой культуры.

Отчество упоминается в русских летописях с XII в. На Руси говорили:\textit{ Как Вас звать-величать?} {\bfseries Величание}, то есть обращение по отчеству, --- это демонстрация уважительного отношения к человеку.
Сначала по отчеству называли князей, затем бояр и дворян.
Существовали также формы полуотчеств со словом сын: \textit{Федор Иванов сын}.
Петр I отличившимся в служении государству людям (например, купцам) жаловал величание по отчеству как знак особого уважения.

Екатерина II повелела особ первых пяти классов "Табели о рангах" писать с - \textbf{(в)ичем}, чинов VI-VIII классов --- с полуотчествами, а всех остальных --- только по имени.
К примеру, профессор Императорского Московского университета в соответствии с этим указом мог удостоиться только полуотчества.
С середины XIX в. все другие сословия (кроме крепостных крестьян) уже пользовались отчеством на -(в)ич, -(в)н-(а).

Отчество человека известно уже при рождении, но входит в употребление по достижении человеком социальной зрелости.
\textbf{Отчество коррелирует с формой обращения на Вы.}

Если личному имени свойствен дейктический (указательный) статус, то \textbf{отчество подчеркивает авторитет личности, а отсутствие отчества --- неуважение к человеку}, ср.: \textit{Иван Иванович сказал, значит надо сделать} и \textit{Да что нам этого Ваньку слушать!}

Сейчас в СМИ формы обращения по имени-отчеству остаются неизменными только в отношении старшего по возрасту, очень уважаемого человека.
Исследования показывают, что, убирая отчество, мы "отчуждаем" человека, переводим общение в сугубо официальную сферу.
Когда человек говорит о своем учителе, родителях, он не может не использовать отчество, но в отчужденном смысле известного человека можно именовать по имени и фамилии: \textbf{ Лев Толстой, Сергей Эйзенштейн, Марина Цветаева.}

СМИ, отучая народ от необходимого русскому человеку отчества, подают {\bfseriesплохой пример отступления от норм русского речевого употребления, нарушают правила речевого этикета и коммуникативного поведения}, потому что отчество является неотъемлемым элементом русского национального менталитета.

\subsection*{Что такое базовая культура личности?}

В последнее время стали говорить уже об \textbf{элементарном уровне культуры личности} --- \textit{базовой культуре личности, то есть необходимом минимуме общих способностей человека, его ценностных представлений и качеств, без которых невозможна как социализация, так и оптимальное развитие генетически заданных дарований личности} (О.~С.~Газман).
Определены и основные компоненты базовой культуры личности: комплекс знаний, умений, качеств, привычек, ценностных ориентаций.
При этом предполагается в первую очередь знание базовых фактов культуры того или иного народа или мировой культуры.

Применительно к речевой культуре особенно это касается знания так называемых прецедентных текстов (\textbf{Библии, мифов, вершин мировой литературы и~т.~д.}), поскольку «\textbf{знание прецедентных текстов есть показатель принадлежности к данной эпохе и ее культуре}, тогла как их незнание, наоборот, есть предпосылка отторженности от соответствующей культуры» (Ю.~Н.~Караулов).

{\bfseriesЧеловек культурный --- это всегда человек образованный}.
Но, по сути, культура предполагает не \textit{обучение}, а \textit{образование} личности.
И оба эти слова не случайно различаются по смыслу.
В этом отношении \textit{человек обученный от человека образованного отличается тем, что он не знания получил, а умение добывать, применять и передавать эти знания}.
Он не чьи-то мысли выучил, а научился сам мыслить.
Он не только может повторить, кто и что сказал и то, как нужно говорить в той или иной ситуации, но сам в состоянии создать свою речь.
Все это можно сформулировать так: {\bfseriesкультурный, образованный человек --- это человек не только много знающий и умеющий, но, прежде всего, творческий}.
Поэтому культура личности обязательно предполагает овладение не только некоторыми необходимыми знаниями, умениями и навыками, но --- {\bfseries личностно освоенными в деятельности культурными ценностями.}
Следовательно, в основе образования лежит процесс познания нахождение взаимосвязей между различными явлениями и их структурирование.
А далее структура подсказывает лакуны, белые пятна, которые тоже требуют наполнения.
Таким образом, познание, как компонент культуры, тоже диалогично, потому что идет движение от незнания к неполному знанию, а затем --- к относительно полному.

Культура в наиболее общем виде предполагает наличие {\bfseriesиндивидуальности} у человека и стремление проявить эту индивидуальность.
И в то же время признания этой индивидуальностью права других людей на их индивидуальность и уважение этой их индивидуальности.
Не обособление личности, а осознание, что индивидуальность --- это отдельность внутри целого --- общества.
Это предполагает умение взглянуть на ситуацию с позиций другого, понять ход его мыслей, его чувства и~т.~д.
Все это проявляется опять-таки в диалогической деятельности, а именно в процессе общения.
\textbf{Общая культура и общества, и коллектива, и человека немыслима без культуры общения.}

Таким образом, \textbf{культура} --- это система достижений человека во всех отраслях жизни, которая появилась и развивается благодаря целенаправленной и осознанной деятельности человека и общества в целом в материальной и духовной сферах.
Эта деятельность носит {\bfseriesобъединяющий коммуникативный характер}, направленный на достижение высшего уровня качества, и потому сопровождается системой ограничений, благодаря которым происходит целенаправленный отбор достойного для продолжения и развития. 

\subsection*{Языковая личность.}

См. 11 билет. TODO: переделать. Никуда не годится. 11-м билетом зимой был "Понятие языковой нормы. Источники кодификации
языка. Типы словарей."
