\section{Коммуникативные неудачи. Способы предотвращения коммуникативных неудач.}

Коммуникативные неудачи постоянны в общении людей, они естественны и часто приводят к недопониманию.
Это неизбежные спутники общения и следствие культурного и языкового барьеров.
Причем культурный барьер опаснее языкового, потому что культурологические ошибки воспринимаются гораздо болезненнее и вызывают более негативную реакцию.
Обычно родная культура вполне естественно воспринимается как единственно правильная.
\textbf{Этноцентризм} - свойство почти всех культур.

Коммуникативные неудачи классифицируются по разным основаниям: социально-культурным, психосоциальным и языковым.

К коммуникативным неудачам нередко приводят различия в речевых стратегиях говорящего.
Нарушение норм национально специфического речевого поведения воспринимается как непроизвольное вторжение в интимную сферу.
Например, в Японии будет странным вопрос в транспорте: \textit{Вы выходите?}
Он считается бестактным, потому что нарушает границы личности.
На подобный вопрос может последовать ответ: \textit{А Вам какое дело? Хочу выхожу, а хочу - нет.}
Японцу надо подать едва заметный сигнал, чтобы он ощутил некоторое неудобство и догадался, как Вам помочь.

Коммуникативные неудачи связаны с недостаточным знанием не только языка, но и обычаев другого народа.
Так, в Китае суп подают после еды; не зная, что это означает завершение трапезы, иностранцы могут затянуть свой визит в ожидании продолжения.
Другой случай: американец, приглашенный в японскую семью на обед, уходя, стал благодарить хозяев.
Оказалось, что это культурологическая ошибка, потому что в Японии в этой ситуации используются не формулы благодарности, а формулы извинения.

Коммуникативные неудачи могут быть связаны с невербальными средствами общения.
С.Г. Тер-Минасова в книге «Война и мир языков и культур» приводит такой пример.
В январе 2005 г. в европейской прессе прошло сообщение: общественность Норвегии была шокирована тем, что во время инаугурации президент США Джорж Буш сделал жест, который у норвежцев считается приветствием дьяволу (выставленный вперед указательный палец и мизинец).
Это свидетельствует о множественности восприятия паралингвистических сигналов, которую необходимо принимать во внимание.

Коммуникативное поведение - это совокупность норм и традиций общения в определенном лингвокультурном сообществе.
В русском общении меньше норм и больше традиций, в западном общении меньше традиций и больше норм.
Поэтому русскому человеку легче овладеть высоконормированной западной моделью, чем западному человеку освоить нечетко очерченные традиции русского общения, являющиеся отражением специфики русской культуры, которую Ю.М. Лотман определил как «бинарную», развивающуюся путем взрывов глобального характера.
Культурные расколы, разломы отражаются в размытости норм русского коммуникативного поведения.
