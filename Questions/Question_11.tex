\section{Источники кодификации языка. Типы словарей.}

Закрепление, фиксация языковых норм называется \textbf{кодификацией}.

\textbf{Источниками кодификации} русского литературного языка являются словари, грамматики и учебники.

Словари бывают самые разнообразные: \textit{толковые, двуязычные, терминологические, словари иностранных слов, устаревших
слов, синонимов, антонимов, омонимов, паронимов, фразеологизмов, словари языка писателей и поэтов} и многие другие.

Первый полный толковый словарь русского языка --- « Словарь живого великорусского языка» в четырех томах --- составил Владимир Иванович Даль в 1863--1866 гг.
Первую систематизированную грамматику русского языка --- «Российскую грамматику» --- написал М.В. Ломоносов в 1755 г.

В настоящее время существует несколько изданий «Академической грамматики русского языка» (1952, 1970, 1982 гг.).
Грамматические правила и упражнения для их усвоения содержатся в учебниках русского языка.

В литературном языке нормализации и кодификации подвергаются все стороны языка.

Нормативное \textbf{произношение} закреплено в орфоэпических словарях.
Один из первых таких словарей --- словарь под редакцией Р.И. Аванесова «Русское литературное произношение и ударение», вышедший в свет в 1954 г.
Произношение - очень важный диагностический фактор происхождения человека, показатель его образовательного и культурного уровня.
В орфоэпических словарях встречаются особые пометки: «не рекоменд.» (\textit{красив\underline{е}е , зв\underline{о}нит}), «старая норма» (\textit{кровот\underline{о}чить}).

Обратите внимание на произношение следующих слов, в которых часто допускаются ошибки: \textit{асимметрИя, вероисповЕдание, договОр, звонИт, Иконопись, исчЕрпать, квартАл, красИвее, кулинАрия, мастерскИ, облегчИтъ, обеспЕчение, украИнский, фенОмен, христианИн, в Яслях}.

\textbf{Лексика} современного литературного языка также нормирована.
В литературном языке не допускается использование просторечных и жаргонных форм, типа \textit{базарить, лоханулся, бабло} и т.п.
Следует принять во внимание, что иногда слова из литературного языка в просторечии начинают употребляться в роли экспрессивных частиц с имплицитным (скрытым) значением: например, слово \textit{блин.}
Наличие таких экспрессивных частиц недопустимо в литературном языке.

Очевидно, что это явление хорошо знакомо носителю русского языка.
В современной лингвистике такие слова называются \textbf{словами-паразитами.}
Они заполняют пустоты, отражают неуверенность говорящего, раздражают слушающего и в то же время выявляют особый психологический склад человека, его потребность в постоянной коммуникативной поддержке со стороны собеседника.

Самые отъявленные словесные паразиты нашего времени ---
\textit{как бы, практически, в принципе, по большому счету, абсолютно,
достаточно.}
Весьма широко паразитируют в современном просторечии слово \textit{типа,} аналогичное по смыслу английскому неопределенному артиклю \textit{а,} и слово \textit{конкретно,} соответствующее определенному артиклю \textit{the.}

У слов-паразитов проявляется одно важное свойство --- тенденция к сжатию, так как частотность их употребления очень высока: \textit{так сказать} $\rightarrow$ [тксать].
По наблюдениям лингвистов, хороший, прочный \textit{паразит} должен быть односложным, тогда он, как сорняк, легко приживается в просторечии.

Слово-паразит может быть \textbf{семантически нагруженным} и доказывать, что говорящий опасается делать окончательные выводы. высказываться определенно, хочет выглядеть скромным.
Психолингвисты считают, что слова-паразиты отражают изменения в национальном менталитете.
В данном случае речь идет о неуверенности, боязни утверждения собственной позиции.

Появляясь в языке неожиданно, слова-паразиты, подобно инфекции, заражают организмы, лишенные \textbf{лингвистического иммунитета,} который вырабатывается внимательным и вдумчивым отношением к слову, постоянным вниманием к форме и содержанию речи.

Задача каждого образованного человека --- укреплять лингвистический иммунитет, развивая языковую интуицию, или языковое чутье, препятствуя проникновению в свою речь слов-паразитов.

В современном русском литературном языке нормализации подвергнуты правила \textbf{словообразования, синтаксис} (правила построения высказывания), \textbf{орфография} (правила написания) и \textbf{пунктуация} (правила постановки знаков препинания).
