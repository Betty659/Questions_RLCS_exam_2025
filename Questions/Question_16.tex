\section{Культура письменной речи.}
    \textit{Про культуру речи см. предыдущий билет}

    \subsection*{Особенности письменной речи}
    Письменная речь – созданная людьми знаковая система с целью фиксации и сохранения устной (звуковой) речи. Фиксация устной речи – основная функция письма: обеспечивает возможность усвоить знания, человеческий опыт, разрывает рамки непосредственного общения, расширяет сферу деятельности и человеческого общения.

\begin{itemize}
    \item Основное свойство письменной речи – \textbf{способность к длительному хранению информации}. Письменная речь развертывается не во временном, а в статическом пространстве, что открывает возможность редактирования текста, шлифовки его.
    
    \item Письменная речь функционирует в книжной сфере, она \textbf{строго кодифицирована}. Последовательное подчинение литературной норме в отборе слов, построении предложений делает письменную речь сложной по форме, полной по содержанию, воспринимаемой за образцовую, правильную речь. И хотя письменная речь уступает по употребительности устной, она обладает \textbf{стилеобразующей функцией}: письменная речь является основной формой существования речи в научном, публицистическом, официально деловом и художественном стилях. В самой письменной форме речевой деятельности находят определенное отражение условия и цель общения (например, художественное произведение или описание научного эксперимента, заявление об отпуске или информационное сообщение в газете). Письменная речь \textbf{ориентирована на восприятие органами зрения}, поэтому обладает четкой структурной и формальной организацией: имеет систему нумерации страниц, деление на разделы, параграфы, систему ссылок, шрифтовые выделения и т.п.
    
    \item Основной единицей письменной речи является \textbf{предложение}, которое выражает сложные логико-смысловые связи посредством синтаксиса – сложные синтаксические конструкции, причастные и деепричастные обороты, распространенные определения, вставные конструкции и т.п. При объединении предложений в абзацы каждое строго связано с предшествующим и последующим контекстом.
\end{itemize}
