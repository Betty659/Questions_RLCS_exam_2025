\section{Слушание как вид речевой деятельности. Условия эффективного слушания}

\textbf{Слушание - вид речевой деятельности, в основе которого лежит восприятие и осмысление звучащих текстов.}

Слушание реализуется прежде всего при непосредственном взаимодействии участников общения, хотя если на слух воспринимаются радио- и телесообщения, то следует говорить об их опосредованном взаимодействии. 
В процессе непосредственного взаимодействия говорящий и слушающий не только слышат, но и видят друг друга, что существенным образом сказывается на характере слушания. 
При непосредственном взаимодействии коммуниканты активно используют средства невербального общения (жесты, мимику, телодвижения).

Слушание представляет собой активный мыслительный процесс, направленный на смысловую обработку текста в ходе его восприятия. 
Результатом слушания является \textit{понимание} (или непонимание) звучащего текста. 
Понимание текста проявляется в способности слушающего осознать его основную мысль, осмыслить понятия, которые в нем раскрываются, в умении свободно и полно изложить его содержание. На основе услышанного, если оно понято и осмыслено, создаются новые высказывания.

\textbf{Обычно выделяют следующие функции слушания: }
\begin{itemize}
	\item \textbf{познавательную} (слушаю, чтобы знать);
	\item \textbf{регулятивную }(слушаю, чтобы научиться что-либо делать, сформировать необходимые умения); 
	\item \textbf{ценностную} или ценностно-ориентационную (слушаю, чтобы получить эстетическое наслаждение, чтобы ориентироваться в мире культурных новостей).
	\item \textbf{реагирующую} (слушаю, чтобы ответить на вопрос прохожего, случайного собеседника и т. п.)
\end{itemize}

Таким образом, с помощью слушания человек получает новую информацию, новые знания, овладевает какими-либо умениями, удовлетворяет свои потребности эстетического характера, адекватно реагирует на вопросы, просьбы, пожелания.

