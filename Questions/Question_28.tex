\section{Публичное выступление. Требования к поведению оратора.}
    \subsection*{Ораторский стиль}
     \hspace{\parindent} Подобно тому, как не сразу формируется индивидуальность стиля писателя,  так же не сразу формируется и стиль оратора. Ораторский стиль — это сложное сплетение знаний, опыта, манеры изложения, степени эмоциональности говорящего.

     Большинство специалистов выделяют три основных стиля:
    \begin{enumerate} 
        \item \textbf{Строго логичный} (внешне спокойный).
        \item \textbf{Эмоционально насыщенный}.
        \item \textbf{Средний} (между спокойным и эмоциональным).
    \end{enumerate}
    Другая классификация:
    \begin{enumerate} 
    \item \textbf{Рационально-логический} тип. 
    
Эмоциональность этих людей внешне практически не
 проявляется, хотя это не значит, что она отсутствует вообще. 
 Они больше склонны к анализу явлений, к рассуждениям, к строгой аргументированности. Их подготовка к любому докладу отличается последовательным отбором и строгой систематизацией материала, обдумыванием и разработкой подробного плана. 
 Этот выношенный план как бы «сидит у них внутри», но, как правило, ораторы этого вида не пользуются им. 
 Чаще их заботит другое: как сделать свою речь яркой, эмоциональной, какой иллюстративный материал подобрать, чтобы заинтересовать аудиторию. 
 Наблюдения показывают, что чаще всего к этому типу принадлежат \textit{сангвиники} — люди с заметной психической
 активностью, быстро реагирующие на происходящие события, стремящиеся к смене впечатлений, не зацикливающиеся на неприятностях и неудачах, живые, подвижные, с выразительной мимикой и жестикуляцией.
 \item \textbf{Эмоционально-интуитивный} тип. 
 
 Люди этого типа говорят страстно, увлеченно, но
 не всегда могут уследить за жесткой логической последовательностью речи. 
Если у них не составлен жесткий план, которому они должны следовать, то их, как правило, «заносит», мысль теряется, эффект блистательной речи сводится на нет. 
У слушателей остается впечатление эмоционального разговора ни о чем. 
Несмотря на это, ораторы, принадлежащие к этому типу, не любят работать над планами, считая, что наличие плана сковывает. 
Считается, что к этому типу ораторов относятся люди с холерическим темпераментом: \textit{холерики} — люди очень энергичные, способные увлекаться, склонные к бурным эмоциональным вспышкам.
\item \textbf{Философский} тип. 

Люди, относящиеся к этой группе, труднее всего поддаются описанию, так как одновременно достаточно эмоциональны и достаточно логичны. 
Их индивидуальность наименее четко выражена, но всех их объединяет одна черта — стремление к исследованию, осмыслению явления прямо на глазах у слушателей, желание и умение вовлечь в этот процесс аудиторию. 
Чаще всего к этой группе относятся люди флегматического темперамента: \textit{флегматик} — человек невозмутимый, с
устойчивым стремлением и настроением, с постоянством чувств, со слабым внешним выражением душевных состояний.
\item \textbf{Лирический, или художественно-образный} тип. 

Этих людей отличает глубокая эмоциональность, лиризм, острая впечатлительность, проникновенность, хотя в частных проявлениях люди этого типа могут существенно отличаться друг от друга.
Чаще всего этот тип имеет в своей основе меланхолический темперамент: \textit{меланхолик} — человек впечатлительный, с глубокими переживаниями, легко ранимый, но внешне слабо реагирующий на окружающее, со сдержанными движениями приглушенностью речи.
    \end{enumerate}
\subsection*{Рекомендации оратору}
\hspace{\parindent}Произвести хорошее первое впечатление. Внешний вид, жесты, манера держаться, улыбка, взгляд и прочее.

Плохо, когда оратор начинает выступление на ходу, еще не подойдя к трибуне. Необходимо сделать паузу, прийти в себя, дать аудитории настроиться. 

Оратору следует установить зрительный контакт с аудиторией, его нельзя терять ни на минуту. Потеря контакта ведет к потере внимания.

Нельзя показывать аудитории свое волнение, ритуал приведения себя в порядок должен быть незаметен для зрителей.

Не рекомендуется принимать закрытых поз, они воспринимаются как недоверие.

Движение рук должны отображать предмет речи говорящего, указывать на что-то или образно иллюстрировать. Жестикуляция должна быть естественная. Жесты волнения (теребить пуговицу) отвлекают аудиторию, их следует избегать.
\subsection*{Полный текст рекомендаций}
\hspace{\parindent}
Всем известно, что первое впечатление самое сильное, оно хорошо запоминается. 
В дальнейшем его можно только корректировать, менять, а это чревато большими усилиями.
Произвести благоприятное впечатление на аудиторию до момента говорения — задача любого  оратора. 
Это благоприятное впечатление складывается и из уверенной, с достоинством манеры держаться, и из проявления доброжелательности с помощью улыбки, взгляда. 
Оратор может быть уверен, что аудитория оценит и фигуру (при деловом общении благоприятное впечатление складывается от фигуры, напоминающей вытянутый прямоугольник с подчеркнутыми углами), и стиль одежды. 
Но если в фигуре мгновенно изменить что-либо трудно, то надо помнить, что и от выбора одежды зависит многое. 

Причем даже знакомая аудитория (студенческая группа) оценит, если ради выступления докладчик, подчеркнув, что относится к происходящему серьезно, чуть изменит манеру одеваться, сделает ее более строгой. 
Кстати, выбор цвета одежды также несет информацию о владельце: в европейской культуре признаком высокого статуса считаются черно-бело-серые оттенки и гамма с преобладанием светлых тонов. 
Чем ярче и насыщенней цвет одежды, тем меньше она подходит для делового общения, тем ниже осознается предполагаемый статус человека, ее выбравшего. 
Конечно, доклад на семинарском занятии — не бог весть какое торжественное событие, но и не самое рядовое, поэтому одежда
должна соответствовать серьезности настроения оратора, так как это настроение, несомненно, передастся и слушателям. 

Плохо, если оратор начинает свое выступление на ходу, когда он еще только подходит к трибуне, кафедре или столу неправильно, если по ходу движения он начинает что-то выяснять у преподавателя или ведущего, пусть это будет вопрос о времени, отведенном для сообщения, или о возможности использования необходимых средств наглядности. Самое целесообразное — задать все эти вопросы заранее. Напротив, необходимо сделать паузу, дать возможность успокоиться самому и сосредоточиться аудитории. 
Кроме того,эти несколько секунд нужны, чтобы слушатели составили о выступающем то самое первое впечатление. Кстати, если нет ни кафедры, ни трибуны, ни стола, говорящему лучше встать прямо перед ними, на расстоянии 2—3 метров от первого ряда. 
Такое расположение докладчика даже предпочтительнее: трибуна сковывает и скрывает жесты, уменьшает возможность эмоционального влияния на аудиторию. 
Во время начальной паузы оратору необходимо установить зрительный контакт с аудиторией, то есть обвести взглядом
собравшихся, посмотреть им в глаза.
Установившийся визуальный контакт нельзя терять ни на минуту. 

Вот в чем одна из причин того, что написанный текст нельзя читать: даже если читающий время от времени поднимает глаза на слушателей, зрительный контакт с аудиторией постоянно прерывается. 
К сожалению, начинающие ораторы, не имея практического навыка поддержания такого контакта, часто отводят глаза, смотрят поверх голов, поднимают глаза к потолку, что мгновенно расхолаживает слушателей, ведет к потере интереса к предмету речи.
Самое правильное — \textit{мысленно разбить всех на группы и переводить взгляд, обязательно фиксируя на несколько секунд, от одной группы к другой}. 
Этой рекомендацией следует руководствоваться непременно, даже если смотреть в глаза слушателям страшновато. 
С фиксацией взгляда связана еще одна очень распространенная ошибка начинающих ораторов: если преподаватель сидит вместе со слушателями, то взгляд говорящего студента, как правило,
обращен к нему.
От преподавателя ждут одобрения, поддержки. 
Такое поведение, вызванное беспокойством и волнением, можно понять, но поступать так нельзя: предоставленная самой себе аудитория сразу выходит из-под контроля. 
Кстати, предваряющая пауза может помочь начинающему оратору справиться с излишним волнением. 

Появление волнения перед лицом слушателей — вещь закономерная, плохо только, если это волнение чрезмерно, если оно парализует способность адекватно мыслить, если во рту появляется сухость, колени дрожат и кажется, что невозможно сдвинуться с места. 
В такой ситуации оратор должен взять себя в руки, сделать несколько глубоких вдохов и выдохов, отвлечься, передвинув какой-либо предмет на столе. 
Вдохи не должны быть очень глубокими, движения — суетливыми, чтобы не выдать волнение залу. 
Потом нужно постараться сосредоточиться на предмете речи, начать говорить — это должно отвлечь, и боязнь постепенно пройдет. 

С преодолением волнения связано решение еще одной проблемы — местоположение оратора в пространстве. 
В обыденной речи человек редко задумывается над тем, как ему встать, как расположить ноги и чем занять руки, а в
минуты волнения эти простые вопросы начинают казаться неразрешимыми. 
Вместе с тем решить эту проблему просто: встать нужно так, как удобно, но обязательно устойчиво. 
Для этого можно чуть выдвинуть вперед ногу, перенести центр тяжести на другую. 
Такая поза даст возможность, перенося центр тяжести, делать шаг вперед — назад, чтобы не стоять перед слушателями неподвижным памятником в течение всего выступления. Известно, что мышечная напряженность и усталость выступающего сразу передается слушателям, они начинают ерзать, шевелиться на своих местах. 
Изредка меняя позу, оратор не даст утомиться аудитории. 
Но здесь нельзя злоупотреблять: если все время менять позу, появится ощущение, что говорящий не находит себе места. Кроме того, этим же движением — шагом вперед — можно подчеркнуть важную мысль, сосредоточить на ней внимание аудитории. 
Оптимальное положение рук следующее: они согнуты в локтях, так что ладони находятся выше уровня талии, пальцы расположены так, будто в руках у говорящего грейпфрут. Такое положение рук легко читается аудиторией как расположение и готовность к общению. Мало того, можно наблюдать интересную вещь: даже если оратор сильно волнуется и немного зажат, то есть к полноценному общению с аудиторией не готов, а, следовательно, описанная выше поза кажется ему несколько неестественной, но он все-таки принимает ее, то через некоторое время такое положение рук
оказывает действие не только на слушателей, но и на самого говорящего — настраивает его на полноценное искреннее общение. 
Эта метаморфоза аналогична ситуации с улыбкой: если настроение плохое, нужно пересилить себя и несколько раз улыбнуться, и хотя сначала улыбка будет вымученной, через несколько минут настроение существенно улучшится и та же улыбка станет искренней и естественной. 
Вместе с этим не рекомендуется принимать закрытых поз, когда скрещены руки и (или) ноги, потому что эта поза воспринимается аудиторией как выражение недоверия, неготовности к общению. 
Отрицательно воспринимается слушателями и поза, при которой говорящий опирается руками о стол, наклонившись над ним — это поза доминирования, превосходства. 
При любой позе особое внимание будет обращено на руки говорящего. 

Жестикуляция — обязательная принадлежность любой устной речи, кроме, пожалуй, строго официальных ситуаций, оговоренных протоколом. 
Движения рук могут подчеркнуть главное (ритмические жесты), направить внимание аудитории на доску или таблицу (указательные жесты), изобразить предмет речи (изобразительные жесты), отреагировать на поведение аудитории (реагирующие жесты). 
Все перечисленные группы жестов должны использоваться в речи естественно, но должны использоваться обязательно, поскольку аудитория не только слушает, но и зрительно воспринимает предлагаемую информацию. 

Эмоциональность и активность жестикулирования зависит от многих причин: эмоциональности самого оратора (понятно, что чем эмоциональнее человек, тем эмоциональнее его жесты, так как жесты есть внешнее выражение внутренней свободы, раскованности), эмоциональности предлагаемого материала (чем эмоциональнее материал, тем эмоциональнее жесты), от национальной принадлежности оратора (традиционно считается, что южане жестикулируют активнее, чем северяне) и ряда других причин. 
Надо помнить, что жест оратора вызывает аналогичные скрытые движения и у слушателей, передают аудитории соответствующие эмоции. 

Кроме ярких, эмоциональных, крайне необходимых и говорящему, и слушателям жестов есть особая группа жестов, производимых непроизвольно из-за очень сильного волнения. Вслед за А. К. Михальской можно назвать их «манеризмами»: оратор теребит пуговицу или бусы, подкручивает часы, ломает пальцы, почесывает ухо — все это устойчивые привычки, от которых очень трудно, но необходимо избавиться. 
Подобные жесты отвлекают аудиторию от содержания речи и дискредитируют оратора, выдают его волнение и неумение с ним справиться. 
Таким образом, жесты, телодвижения, равно как и мимика (доброжелательное, увлеченное, но не сердитое или равнодушное лицо), не являются самоцелью. 
Действуя на зрительный канал восприятия, они подчеркивают наиболее важную информацию, повышают эмоциональность речи и способствуют более эффективному ее усвоению. 
Аналогична функция интонационного рисунка речи, голосовых характеристик, дикции.
\subsection*{Действия, способствующие поддержанию внимания:}
\begin{itemize}[noitemsep] 
\item логически стройное изложение материала;
\item сохранение интригующих моментов или конфликтов;
\item обращение к средствам наглядности (примеры из жизни, сравнения, аналогии, отсылки на произведения и прочее);
\item речевая динамичность (голос, интонация, темп, тон и прочее);
\item языковые средства поддержания внимания (повторения, риторические вопросы, обращение к аудитории и прочее);
\item диалог с аудиторией;
\item краткие отвлечения от темы (юмор, истории);
\item лаконичность (краткость) не в ущерб содержательности.
\end{itemize}
